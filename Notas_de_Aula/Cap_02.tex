
\chapter{Comando locate}


\section{Introdução}

Quando é preciso localizar alguns arquivos no sistema ou em alguns diretórios, 
pode-se usar o comando \texttt{find} para encontra-los. Embora ele seja 
um bom utilitário para realizar pesquisas, porém ele é lento.

No entanto o comando \texttt{locate} pode procurar arquivos com muita rapidez. 
Embora o comando \texttt{locate} seja muito rápido, ele ainda não permite
que se deixe de lado o comando \texttt{find} porque ele tem algumas limitações,
como será mostrado.

\section{Como o comando \texttt{locate} funciona? -- updatedb e updatedb.conf}


Quando foi dito que o comando \texttt{locate} faz pesquisas muito rapidamente, 
a primeira questão que surge é o que o comando \texttt{locate} faz para ser tão 
rápido?

Bem, o comando \texttt{locate} não busca os arquivos no disco, em vez disso, 
ele procura pelos arquivos em caminhos definidos em um banco de dados ``database''. 
O banco de dados ``database'' é um arquivo que contém as informações sobre todos
os arquivos do seu sistema e seus respectivos caminhos. 

O arquivo de banco de dados ``database'' do comando \texttt{locate} está 
localizada em:


A próxima questão lógica é: o que mantém esta base de dados \texttt{mlocate.db} 
do comando \texttt{locate} atualizada?

Bom o utilitário responsável por essa tarefa é o \texttt{updatedb}, o qual 
quando o mesmo é executado, ele verifica todo o sistema e atualiza o arquivo 
do banco de dados \texttt{mlocate.db}. Uma da limitação do comando \texttt{locate}
é a sua dependência em relação ao banco de dados que pode ser atualizado 
pelo utilitário \texttt{updatedb}. Portanto, para obter resultados confiáveis e
atualizados em sua pesquisa com o comando \texttt{locate}, o banco de dados 
que ele usa para realizar a pesquisa deve estar sempre atualizado e para tal 
é necessário atualizar o banco de dados \texttt{mlocate.db} com o comando 
\texttt{updatedb} em intervalos regulares.

Pode-se configurar o utilitário \texttt{updatedb} conforme suas necessidades. 
Isto pode ser conseguido através da atualização do \texttt{updatedb.conf}. Este é um 
arquivo de configuração que \texttt{updatedb} lê antes de atualizar o banco de dados. 
O \texttt{updatedb.conf} está localizado em \texttt{/etc/}:
