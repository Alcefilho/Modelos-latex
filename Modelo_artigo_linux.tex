\documentclass[a4paper,12pt,portuguese,oneside,final,notitlepage]{article}
% =============>  Fontes  <========================
\usepackage{ucs}
%\usepackage[latin1]{inputenc}  % Antiga codificação padrão
\usepackage[utf8]{inputenc}     % Atualmente é a codificação padrão
%\usepackage{indentfirst}       % indenta os primeiros parágrafos
\usepackage[T1]{fontenc}
% Altera estrutura do documento
\usepackage[text={17cm,25.0cm},centering]{geometry}
\usepackage{caption}
\usepackage[brazil,brazilian,english,USenglish]{babel}
% Referências
%\usepackage[num]{abntcite}
%\usepackage[alf]{abntcite}
%\usepackage[num]{abntex2cite}
\usepackage{cite}
% Define macros para escrita de equações
\usepackage{amsmath}
\usepackage{amsfonts}
\usepackage{amssymb}
\usepackage{amsthm}
\usepackage{enumerate}
% Outros
\usepackage{multicol}
\usepackage{rotating}
\usepackage{subfigure}
\usepackage{xcolor}
\usepackage{graphicx}
% Para obter um documento com hyperlinks
\usepackage{hyperref}
%---------------------------------------------------------------------
 \hypersetup{
    bookmarks=,            % Mostra os bookmarks
    bookmarksnumbered=,    % Numera o bookmarks
    pdfborder={0 0 0},     % Sem caixa em volta da hiperreferência
    pdftoolbar=true,       % Mostra a barra de ferramentas no Acrobat Reader = true ==> sim
    pdfmenubar=true,       % Mostra o menu no Acrobat Reader = true ==> sim
    pdffitwindow=false,    % Ajusta o documento ao tamanho da janela aberta = false==> não
    pdfstartview={FitH},   % Ajusta a largura da página a da janela aberta 
    pdfauthor={Salviano A. Le\~{a}o},  % Autor do documento
    pdftitle={Modelo de artigo},       % Titulo do documento
    pdfsubject={Modelo},               % Assunto do documento
    pdfkeywords={Modelo,artigo},       % Palavras chaves do documento
    pdfproducer={LaTeX},               % O produtor do documento
    pdfcreator={Salviano A. Le\~{a}o}, % O programa ou quem criou o documento
    pdfnewwindow=true,   % Abre links em uma nova janela
    colorlinks=true,     % false: links em uma caixa; true: links coloridos
    linkcolor=red,       % Cor dos links internos (mude a cor da caixa com linkbordercolor)
    anchorcolor=black,   % Cor do texto
    citecolor=green,     % Cor dos links das referências bibliográficas
    filecolor=magenta,   % Cor dos links dos arquivos
    urlcolor=cyan        % Cor dos links externos
}
%Note que o último item não tem vírgula.
%--------------------------------------------------------------------
% Definição de algumas funções matemáticas
%====================================================================
\def\limfunc#1{\mathop{\rm #1}}
\def\func#1{\mathop{\rm #1}\nolimits}
\def\unit#1{\mathop{\rm #1}\nolimits}

%%====================================================================

\definecolor{Cinza50}{gray}{0.50}
\definecolor{Cinza10}{gray}{0.10}
\definecolor{Cinza}{gray}{0.80}
\definecolor{cinza}{rgb}{0.9,0.9,0.9}
%-----------------------------------------------------------------------------------

%\newcounter{nota}
\newcommand{\Nota}[1]{%
\begin{center}
   \colorbox{Azul}{\parbox{0.8\linewidth}{\textbf{NOTA:~} #1}}
\end{center}}%

%-----------------------------------------------------------------------------------
\usepackage{listings}
\lstnewenvironment{code}{%
\lstset{%
frame=single,
escapeinside=~~,
language=bash,
inputencoding=utf8,
stringstyle=\footnotesize\ttfamily,
showstringspaces=false,
extendedchars=true,
upquote= true,
keywordstyle=\color{black}\bfseries,
backgroundcolor=\color{Cinza},
basicstyle=\footnotesize\ttfamily}
}{}

% \lstnewenvironment{shell}{%
% \lstset{%
% frame=single,
% escapeinside=~~,
% language=bash,
% inputencoding=utf8,
% stringstyle=\footnotesize\ttfamily,
% showstringspaces=false,
% extendedchars=true,
% upquote= true,
% backgroundcolor=\color{Cinza},
% keywordstyle=\color{blue}\bfseries,
% basicstyle=\footnotesize\ttfamily}
% }{}


\title{\textbf{Cuidados com a codificação}}
\author{Salviano A. Leão}
\date{06-12-2013}

\begin{document}

\maketitle

\section{Introdução}

Nos sistemas Linux, as terminações de um arquivo servem basicamente
para o usuário, pois o sistema,


Em sistemas Linux, encontramos diversos tipos de arquivos, geralmente 
identificados por suas terminações: 

Esta é a convenção, que nem sempre é usada. Pode-se perfeitamente criar 
um arquivo texto que não tenha a terminação .txt. O comando `file' nos permite 
identificar a que categoria um arquivo pertence. Vejamos alguns exemplos: 



\begin{thebibliography}{99}

   \bibitem{Ref01} \url{http://www.leg.ufpr.br/doku.php/dicas:caracteres}

   \bibitem{Ref02} \url{http://www.dicas-l.com.br/arquivo/gnu_linux_tipos_de_arquivos.php#.UqMvj0Mdx38}

   \bibitem{Ref03} \url{http://latexbr.blogspot.com.br/2011/01/recodificando-seus-arquivos-iso-para.html}
   
   \bibitem{Ref04} \url{http://sergioaraujo.pbworks.com/w/page/15863917/iconv}
   
   \bibitem{Ref05} \url{http://cosaslibres.com.co/ayuda-en-linux/man-comandos-linux/comando-file-linux/}
   
   \bibitem{Ref06} \url{http://pt.wikipedia.org/wiki/File_%28Unix%29}
   
   \bibitem{Ref07} \url{http://pt.wikipedia.org/wiki/N%C3%BAmero_m%C3%A1gico_%28inform%C3%A1tica%29}

   \bibitem{Ref08} Neste síto temos alguns comandos linux:
   \url{http://www.devin.com.br/comandos-manipulacao-de-arquivos/}

   \bibitem{Ref09} Neste síto temos alguns comandos linux:
   \url{http://www.vivaolinux.com.br/dica/Principais-comandos-do-Linux}

   \bibitem{Ref10} Neste síto temos alguns comandos linux:
   \url{http://pt.wikipedia.org/wiki/Anexo:Utilit%C3%A1rios_de_Linux/Unix}

   \bibitem{Ref11} Neste síto temos alguns comandos linux:
   \url{http://www.pixelbeat.org/cmdline.html}
   
   \bibitem{Ref12} Neste síto temos alguns comandos linux:
   \url{http://pt.wikibooks.org/wiki/Guia_do_Linux/Iniciante+Intermedi%C3%A1rio/Comandos_diversos}
   
   \bibitem{Ref13} Neste síto temos alguns comandos linux:
   \url{http://www.gnu.org/software/coreutils/manual/coreutils.html}
   
   \bibitem{Ref14} Neste síto temos alguns comandos linux:
   \url{http://caradoacre.wordpress.com/2009/10/17/pequenos-comandos-que-podem-salvar-o-seu-dia/}

   \bibitem{Ref15} Neste síto temos alguns comandos linux:
   \url{http://stackoverflow.com/questions/704130/can-i-transpose-a-file-in-vim}

   \bibitem{Ref16} Neste síto temos alguns comandos linux:
   \url{http://stackoverflow.com/questions/1729824/transpose-a-file-in-bash}
   
   \bibitem{Ref17} Neste síto temos alguns exemplos do tikz:
   \url{http://www.texample.net/tikz/examples/} ou usar \href{http://www.texample.net/tikz/examples/}{Tikz}
\end{thebibliography}

\end{document}
