\documentclass[12pt]{article}

% The packages amsmath and amssymb are developed by The American
% Mathematical Society.
%The American Mathematical Society has developed a set of fonts and macros 
%for specialized mathematical typesetting, above and beyond the math fonts and
%capabilities designed by Donald Knuth for TeX and by Leslie Lamport for \LaTeX\,. The AMS macros have special theorem environments and math environments
%and support for matrices.

% graphicx --- The graphicx package implements \LaTeX\, support for including 
%graphics files, rotating parts of a page, and scaling parts of a page. 

% multirow --- The multirow package automates the procedure of constructing
%              tables with columns spanning several rows by using the 
%              \multirow command. 

% setspace --- The setspace package replaces the doublespace package. 
%It produces double and one-and-one-half line spacing based on the point 
%size in use. The default is single spacing. When you add the setspace package,
% you can set the overall document spacing and the spacing for portions of the
% document. Changing spacing in the entire
%document doesn't affect footnotes, for which you must specify spacing 
%individually. In addition to the package commands, a linespacing option is 
%available through the Options and Packages command on the Typeset menu. 

\usepackage{epsfig,amsmath,amssymb,graphicx,multirow,setspace}

%\pdfcompresslevel=9

% the following produces 1 inch margins all around with no header or footer
\marginparwidth 0pt
\oddsidemargin  0pt
\evensidemargin  0pt
\marginparsep 0pt
\topmargin   0pt
\textwidth   6.5in
\textheight  8.30in                
\footskip 25pt
\voffset -0.5in

\setstretch{1} % The setstretch in double spacing is normally set to "2." 
%If you change it to "1," you will be
%back into single spacing again. 

\makeatletter  %make @ as letter in the following command
\renewcommand{\@oddhead}
{\vbox{\hbox to \textwidth{\strut\rm{Intermediate \LaTeX\,\hfill\rightmark \thepage}}\hrule}} 

\newcommand{\slope}[1]{{\beta_{#1}}}
\newcommand{\ub}[1]{{\boldsymbol{#1}}}   
\def\wCov{{\widehat{{\rm Cov}}}}
\def\logit{{{\rm logit}}}
\def\N{{{\rm N}}}
\def\slopeTone{{\beta_{T1}}}
\def\slopeTtwo{{\beta_{T2}}}             

%\def is a Tex command. For efficiency, most \LaTeX\, commands are defined with
%TeX's \def command. However, the \LaTeX\, commands are safer.

%\newcommand \renewcommand

%\newcommand{cmd}[args][opt]{def}
%\renewcommand{cmd}[args][opt]{def}
%\providecommand{cmd}[args][opt]{def} -- \LaTeX\,2e

%These commands define (or redefine) a command.
%
%       cmd - The name of the new or redefined command. A \ followed by a string of lower and/or uppercase letters or a \ followed by a
%       single non-letter. For \newcommand the name must not be already defined and must not begin with \end; for \renewcommand it must already
%       be defined. The \providecommand command is identical to the \newcommand command if a command with this name does not exist; if it
%       does already exist, the \providecommand does nothing and the old definition remains in effect. 
%
%       args - An integer from 1 to 9 denoting the number of arguments of the command being defined. The default is for the command
%       to have no arguments. 
%
%       opt - (\LaTeX\,2e only) If present, then the first of the number of arguments specified by args is optional with a default value of opt;
%       if absent, then all of the arguments are required. 
%
%       def - The text to be substituted for every occurrence of cmd; a parameter of the form #n in cmd is replaced by the text of the nth
%       argument when this substitution takes place. 
%
%
%
\begin{document}

\pagenumbering{roman} % Roman numerals

\title{ \vspace{1.5in} Intermediate \LaTeX\,} % the document title

\author{Weiliang Qiu\thanks{supported by UBC Stats Dept.}\\6356 Agricultural Road\\
     %   University of British Columbia\\
        UBC \\
        Vancouver BC \\
        V6T 1Z2 
        \and Eugenia Yu \\ 
        6356 Agricultural Road\\
     %   University of British Columbia\\
        UBC \\
        Vancouver BC \\
        V6T 1Z2}

\date{\today}
\maketitle


% To eliminate page numbers (and headers/footers) use \pagestyle{empty} in
% preamble or \thispagestyle{empty} in text for single pages.

\thispagestyle{empty} 

% To set the page number, e.g. to start the paper on p.237, you can use
% \setcounter{page}{237} in the preamble (or in the text). 
% The following command means to start the page on p.0. 
\setcounter{page}{0}            
\newpage

\tableofcontents  %create table of contents automatically
\listoffigures % produces a list of figures automatically
\listoftables % produces a list of tables automatically

\newpage % start a new page

\begin{abstract}
This is the abstract ...
\end{abstract}

\pagenumbering{arabic}     % Arabic page numbers from now on

\section{Introduction}
\LaTeX\, is a document processing system in which you specify the content
and layout(style) separately. That is, a \LaTeX\, file includes commands that
define the structure, and the process requires a compiler to format the final
result.

The basic structure and principles of a \LaTeX\, document will be given in
Sections \ref{structure} and \ref{principles}.
Section \ref{include} shows how to split a larger document into smaller
parts for easier file management.
Special symbols, fonts and lists will be discussed in Sections
\ref{special} to \ref{lists}.
We will talk about how to type math, create tables and insert figures
in Sections \ref{math} and \ref{tables}.
This document ends with examples of handling errors in Section \ref{errors}.


\section{The Edit/Compile/Preview Process with Bibliography}
\label{Processing}

\begin{enumerate}
  \item 
    Use any text editor (e.g. gvim, emacs) to input bibliography 
    information in the file, e.g. mybib.bib.

  \item 
    Use any text editor (e.g. gvim, emacs) to input content and \LaTeX\, 
    commands in the file, e.g. sample.tex.
    Include the following latex commands just before
      \verb1\end{document}1:
      \begin{verbatim}
      \bibliographystyle{plain}
      \bibliography{mybib}
      \end{verbatim}

  \item Compile the \LaTeX\, file by using the command: {\bf latex sample.tex}
  \item Compile the bib file by using the command: {\bf bibtex sample}
  \item Compile again the \LaTeX\, file by using the command: {\bf latex sample.tex}
  \item Compile once more the \LaTeX\, file by using the command: {\bf latex sample.tex}
  \item Display the sample.dvi file by the command: {\bf xdvi sample}
  \item Convert the dvi file to a postscript file with the command:
        {\bf dvips -o psfilename.ps sample.dvi}
  \item Print the sample.dvi file by the Unix command:
        {\bf dviprint sample} or {\bf dvips sample.dvi \verb1|1 lpr -d sample.dvi}

\end{enumerate}

% The symbol % is used to start a comment.

% \verb command simulates a short piece of typed text inside an ordinary
% paragraph. Its argument is enclosed by a pair of
% identical characters. The argument of the following \verb command is
% contained between the two 1 characters. Instead of 1, you can use any
% character that does not appear in the argument except a space, a letter,
% or a *.

The commands {\it latex sample.tex} and {\it bibtex sample} produce a number of files including:
\begin{description}
  \item[example.aux] Used for cross-referencing and in compiling the table of
                     contents, list of figures, and list of tables.

  \item[example.log] Contains everything printed on the terminal when \LaTeX\,
                     is executed, plus additional information and some extra 
                     blank lines.

  \item[example.dvi] This is the DeVice Independent (DVI) format output of
                     what \LaTeX\, thinks your document should look like.

  \item[example.bbl] This file is written by BibTeX, not by \LaTeX\,, using
                     information on the {\it aux} file. It is read by the 
                     \verb1\bibliography1 command.

\end{description}

\section{\LaTeX\, Document Structure}
\label{structure}

% The verbatim environment allows you to type the text exactly the way you
% want it to appear in the document.

\begin{verbatim}
  \documentclass{article} 
         preamble 
         \begin{document} 
         body 
         \end{document} 
\end{verbatim}


\section{General Principles}
\label{principles}

\begin{enumerate}
  \item
   % The symbols ` and ' create `Quotes' or ``Double quotes''
   All input, both text and formatting commands is in ``ASCII'' text.      
  \item
   Spaces and line breaks are not important. A blank line starts 
   a new paragraph, however. 

%Words are separated by one or more spaces. Paragraphs are separated by one
%or more blank lines. The output is not affected by adding extra spaces or
%extra blank lines to the input file. 
%Use a double backslash to break a line in a specific place.

  \item 
   All commands start with a backslash:\\
   e.g. \verb1\documentclass1
  \item
   Braces are used for ``arguments'':\\
   e.g. \verb1\begin{document}1
  \item
   Square brackets are used for ``optional arguments'':\\
   e.g. \verb2\documentclass[11pt]{article}2
  \item
   Commands are case sensitive.\\
   e.g. \verb1\documentclass1 but not \verb1\DocumentClass1
\end{enumerate}


\section{Splitting the Source File into Parts}
\label{include}
\LaTeX\, source documents can be split into several parts by using 
the \verb1\includeonly1 and \verb1\include1 commands.

% The \include command is used to insert part of the latex code (saved in
% separate tex files) into the source code.
% The command takes the name of the tex file that contains the code to
% be inserted as the argument. Do not include ".tex" in the argument.
% Put the \include command at the place where you want to insert the
% file, and the files will be inserted in the same order as appeared in
% the source code.

% The \includeonly command takes the list of files to be included (separated
% by commas) as the argument.
% Similar to the \include command, only the names without the .tex extension
% are required in the list.
% The command must appear in the preamble before the \begin{document} statement.
% Only the files listed in the \includeonly command will be inserted in the
% compilation of the source code, even if there is 
% \include{filename-NOT-in-includeonly-list} 
% in the source code. 
% The \includeonly command is optional. If not used, the compilation of
% the source code will insert all the files included as arguments of the 
% \include command.

\begin{verbatim}
  \documentclass{article} 
          :
          \includeonly{file1,file2,file3}
          :  
         \begin{document} 
            :
            :
            \include{file1}
            \include{file2}
            \include{file3}
            : 
         \end{document} 
\end{verbatim}

In every of the included files ({\it file1.tex, file2.tex, file3.tex}),
do not include \verb1\documentclass{}1, preamble, \verb1\begin{document}1
and \verb1\end{document}1.

\section{Special Symbols}
\label{special}
Some text characters must be generated by control sequences (i.e., quotes, \verb1{}, [], \1, etc.).
The special symbols include
\verb1$ & % # _ { } ~ ^ \1.
Use a backslash in front of the symbols to correctly display the first seven symbols.
in the above list
To display the last three, use \verb1\verb1 command, e.g. \verb1\verb2\21.
The double backslash \verb1\\1 is used to start a new line.



\section{Fonts}
\label{fonts}
Since \LaTeX\, is a formatter, all changes in the format of text must be explicitly expressed.
Table \ref{font_table} shows the different font 
types\footnote{The font effect might depend on the type size of the document.}.

\begin{table}
\caption{Font Effects.}
\label{font_table}
\begin{center}
 \begin{tabular}{|ll|} \hline
  What you type                                    &   What you see \\     
 \hline
  \verb1{\bf hello}1                               &    {\bf hello}    \\                                   
  \verb1{\em hello} is the same as {\it hello}1    &    {\em hello} is the same as {\it hello}\\
  \verb1{\underline {hello}}1                      &    {\underline {hello}} \\
 \hline
  \verb1{\tiny hello}1                             &    {\tiny hello} \\ 
  \verb1{\scriptsize hello}1                       &    {\scriptsize hello}\\ 
  \verb1{\footnotesize hello}1                     &    {\footnotesize hello}\\ 
  \verb1{\small hello}1                            &    {\small hello}\\ 
  \verb1{\normalsize hello}1                       &    {\normalsize hello}\\ 
  \verb1{\large hello}1                            &    {\large hello}\\ 
  \verb1{\Large hello}1                            &    {\Large hello}\\ 
  \verb1{\LARGE hello}1                            &    {\LARGE hello}\\ 
  \verb1{\huge hello}1                             &    {\huge hello}\\ 
  \verb1{\Huge hello}1                             &    {\Huge hello} \\  
 \hline 
 \end{tabular}
\end{center}
\end{table}

\section{Lists}
\label{lists}

There are basically three types of lists in \LaTeX\,:
\begin{itemize}
   \item    itemization (``bullets'') - \verb1\begin{itemize}1
   \item    enumeration (1, 2, 3, ...) - \verb1\begin{enumerate}1
   \item    description - \verb1\begin{description}1
\end{itemize}

All lists in \LaTeX\, have the same general format:
\begin{verbatim}
       \begin{list-type}
         \item list-entry
         \item next-list-entry
       \end{list-type}
\end{verbatim}

\subsection{The Itemize Environment}
One can have lists within lists by using the \verb1\subitem1 or nested
itemize environments.

 \begin{verbatim}
  \begin{itemize}
    \item
     item one
     \subitem
      subitem one
     \subitem
      subitem two
   \item  
     item two
  \end{itemize}
  
  \begin{itemize}
    \item 
     item one
     \begin{itemize}
      \item
      subitem 
    \end{itemize}
    \item 
     item two
  \end{itemize}
 \end{verbatim}

\subsection{The Enumerate Environment}
% example: line 85
 \begin{verbatim}
  \begin{enumerate}
    \item 
     item one
    \item 
     item two
  \end{enumerate}
 \end{verbatim}

\subsection{The Description Environment}
Description lists are similar to enumerated or itemized lists.
In a description list, each item has both a term and a description.
% example: line 38
 \begin{verbatim}
  \begin{description}
    \item[Step 1]
      Step 1 of the algorithm
    \item[Step 2]
      Step 2 of the algorithm
  \end{description}
 \end{verbatim}



\section{Typing Math in \LaTeX\,}
\label{math}
To type math expressions in the running text, use either of the two
short forms \verb1\(...)\1 or \verb1$...$1,
e.g.  \verb1$\hat{\beta}$1.

One can also use the {\bf displaymath}, {\bf equation} and {\bf eqnarray}
environments:

\begin{enumerate}

  \item
  {\bf displaymath}: \\\\
  e.g.
  \begin{displaymath}
  {\rm logit}(p) = \log(\frac{p}{1-p})  \hspace{0.5in} \mbox{is the definition of logit of $p$}
  \end{displaymath}

  \item
  {\bf equation}: \\\\
  e.g.
  \begin{equation}
    \bar{X} = \frac{\sum_{i=1}^n X_i}{n}
  \end{equation}
  
  The {\bf equation*} environment is the same as the {\bf equation} environment except it
  does not generate equation numbers. \\\\
  e.g.
  \begin{equation*}
  Y_{ij} = \mu + \alpha_i + \beta_j + \epsilon_{ij}, \hspace{0.5in} \epsilon_{ij}\sim N(0,\sigma^2)
  \end{equation*}

  Big delimiters are most often used with array. \\\\
  e.g.
  \begin{equation}
   \left(\begin{array}{c}
   y\\
   z
   \end{array}\right)
  \end{equation}
  

  The \verb1\left1 and \verb1\right1 commands must come in matching pairs, but
  the matching delimiters need not be the same. \\\\
  e.g.
  \begin{equation}
  \left.\frac{\partial f(x, y)}{\partial x}\right|_{x=x_0}=y
  \end{equation}

  We can type text in math mode by using the \verb1\mbox1 command. \\\\
  e.g.
  \begin{equation}
   x=\left\{ \begin{array}{ll}
   y& \mbox{if $y>0$}\\
   z+y & \mbox{otherwise}
   \end{array}\right.
  \end{equation}

\newpage
  \item
  {\bf eqnarray}: \\\\
  The {\bf displaymath} and {\bf equation} environments make one-line
  formulas. A sequence of equations or inequalities is displayed with the
  {\bf eqnarray} environment. By default, an equation number is put in
  every single line. Use the \verb1\nonumber1 command in each of the lines
  you want to suppress an equation number. \\\\
  e.g.
  \begin{eqnarray}
   x&\ll &y_{1}+\cdots +y_{n}\\
   &\leq &z \nonumber
  \end{eqnarray}
  
  The {\bf eqnarray*} environment is the same as the {\bf eqnarray}
  environment except it does not generate equation numbers. \\\\
  e.g.
  \begin{eqnarray*}
   &A\stackrel{a'}{\to}B\stackrel{b'}{\to}C\\
   &\vec{x}\stackrel{\rm def}{=} (x_1,\ldots, x_n)
  \end{eqnarray*}
  
  For multiline formula, sometimes it is desirable to have one equation number
  and put the equation number in the middle line of the equations. In this
  case, we can combine the {\bf equation} environment and the {\bf split}
  environment. \\\\
  e.g.
  \begin{equation}
   \begin{split}
    H_0 : \mu_1 &= \mu_2 \\
    H_1: \mu_1 &\neq \mu_2
   \end{split}
  \end{equation}
  
  We cannot use the \verb1\bf1 command in math mode. Instead we use
  \verb1\mathbf1 for English characters and \verb1\boldsymbol1 for both English
  characters and Greek letters. \\\\
  e.g.
  \begin{equation}
    \Biggl\{
      \begin{array}{c}
       {\mathbf Y_i} = {\mathbf X_i} \boldsymbol{\beta} + {\mathbf Z_i} \boldsymbol{b_i} + \boldsymbol{e_i} \\ 
       \boldsymbol{b_i} \sim {\rm N}(0,{\mathbf D}), \boldsymbol{e_i} \sim {\rm N}(0,{\mathbf R_i}),
             \boldsymbol{b_i} \perp \boldsymbol{e_i}
      \end{array}
  \end{equation}
  
  Sometimes it might be useful to use the {\bf align}, {\bf gather} and {\bf multline}
  environments.
%\newpage
\item {\bf align}:
\begin{align}
x+y&<2\\
x+y&>3\\
x+y&\leq 15\\
x+y&\geq 20000000000000000000
\end{align}
Comparing to the result from using {\bf eqnarray}:
\begin{eqnarray}
x+y&<&2\\
x+y&>&3\\
x+y&\leq& 15\\
x+y&\geq& 20000000000000000000
\end{eqnarray}
\item {\bf gather}:
\begin{gather}
x+y<2\\
x+y>3\\
x+y\leq 15\\
x+y\geq 20000000000000000000
\end{gather}
\item {\bf multline}:
\begin{multline}
x+y<2\\
x+y>3\\
x+y\leq 15\\
x+y\geq 20000000000000000000
\end{multline}
Comparing to the result from using {\bf split}:
\begin{equation}
\begin{split}
x+y&<2\\
x+y&>3\\
x+y&\leq 15\\
x+y&\geq 20000000000000000000
\end{split}
\end{equation}

\end{enumerate}



\section{Tables and Figures}
\label{tables}
To create tables, you will need the {\bf table} and the {\bf tabular} environments.
Table \ref{anova_table} gives the anova table of a 1-way anova, and Table
\ref{multtable} shows an example of using \verb1\multicolumn1 and
\verb1\multirow1 commands.

\begin{table}
\caption{Anova Table.}
\label{anova_table}
\begin{center}
  \begin{tabular}{|c||c|c|c|c|}\hline
    Source of Variation & SS  & df & MS  & F-ratio \\
  \hline
    Treatment           & SST & 2  & MST & $\frac{MST}{MSE}$ \\
    Error               & SSE & 30 & MSE &  \\
  \hline
  \end{tabular}
\end{center}
\end{table}

%r - right justify the column 
%l - left justify the column 
%c - center the column 

\begin{table}[t]
\caption{A sample table with the use of multicolumn and
multirow commands.}
\label{multtable}
\begin{center}
\begin{tabular}{|cl||cc|cc|}
 \hline
  \multicolumn{6}{|c|}{$\slope{C}$ = -0.001721} \\
 \hline
  &&
  %\multicolumn{2}{|c||}{Parameter values used for simulation} &
  \multicolumn{2}{c|}{Method \# 1} &
  \multicolumn{2}{c|}{Method \# 2} \\
 \cline{3-6}
  \multicolumn{2}{|c||}{Parameter values used for simulation} &
  $\hat{\beta}_{time \times group}$  &  ($SE$)&
  $\hat{\beta}_4+\hat{\beta}_5$ & ($SE$)\\
 
  $\slopeTone$ & $\slopeTtwo$ &
   \multicolumn{2}{c|}{(values in $10^{-4}$)} &
   \multicolumn{2}{c|}{(values in $10^{-4}$)} \\
 \hline
 \hspace*{0.25in} \multirow{3}{17mm}{-0.001721}\hspace*{0.25in} & -0.001  & 7.71 & 3.05 & 7.38 & 2.78 \\
                      & -0.0014  &  3.56 & 2.83 & 3.21 & 2.73\\
                      & -0.001721 & 0.45 & 2.99 & 0.07 & 2.70\\
 \hline                                                                                       
\end{tabular}
\end{center}
\end{table}

To insert graphics into \LaTeX, use the {\bf figure} environment.
Figure \ref{hist} shows the histogram of a random sample of 100 observations
from the standard normal distribution.
\begin{figure}
\caption{Histogram of 100 standard normal random variables.}
\label{hist}
  %\includegraphics[width=6.0in]{hist.pdf}
  \epsfig{file=hist.eps,width=6.0in}
\end{figure}


\section{Centering and ``Flushing"}\label{justification}
There are several \LaTeX\, environments to control
the alignment of text.
\begin{verbatim}
     Center: \begin{center} ... \end{center}
     Right Align: \begin{flushright} ... \end{flushright}
     Left Align: \begin{flushleft} ... \end{flushleft}
\end{verbatim}

%\begin{flushright}
The following declarations can be used at the beginning of the body to
produce alignment effect for the entire document.
Be sure to put the declaration immediately after the
\verb1\begin{document}1 command.
These declarations will be turned off as soon as they
encounter a \verb1\end{...}1 command.
%\end{flushright}

\begin{verbatim}
     \centering
     \raggedleft
     \raggedright
\end{verbatim}



\section{Errors in Running \LaTeX\,}
\label{errors}
When you compile a \LaTeX\, file that contains syntax errors, \LaTeX\, will
print out error messages, indicate which line contains an error,
and print a ``?'' prompt. You can either enter letter ``x'' to exit
(quit the compilation) or enter ``E'' to edit the \LaTeX\, file.

%  $x+y

\vskip 1cm
\appendix
\begin{center}
{\bf\LARGE Appendices}
\end{center}
\section{Appendix 1}
 This is the appendix.

For further information on \LaTeX\,, please refer to 
(Lamport, 1986)\nocite{Lamport:1986} or
\cite{Goossens:1994, Lamport:1994}.
%\nocite{Goossens:1994}

\section{Appendix 2}

Also there are many web sites providing useful information on \LaTeX\,. For
example
\begin{enumerate}
  \item A short math guide for \LaTeX: \\
        \verb1http://www.ams.org/tex/short-math-guide.html1 
  \item The \LaTeX\, homepage: \\
        \verb1http://www.latex-project.org/1
  \item Hypertext Help with \LaTeX: \\
        \verb1http://www.giss.nasa.gov/latex/1
\end{enumerate}
\bibliographystyle{abbrv}
\bibliography{ws2}   

\end{document}
