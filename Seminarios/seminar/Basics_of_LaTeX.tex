\documentclass[12pt]{article}
\usepackage{amsmath,graphicx}

%\pdfcompresslevel=9
\begin{document}
% The symbol % is used to start a comment.
% All text after the percentage sign in the same line will be commented.

\title{Basics of \LaTeX} % the document title

% the author
\author{Latex Smart \\
         6356 Agricultural Road\\
         University of British Columbia\\
         Vancouver BC \\
         V6T 1Z2}  

% the date
\date{\today}
\maketitle


\section{Introduction}
\LaTeX\, is a document processing system in which you specify the content 
and layout(style) separately. That is, a \LaTeX\, file includes commands that
define the structure, and the process requires a compiler to format the final
result.

\section{The Edit/Compilation/Preview Process}
\begin{enumerate}
  \item Use any text editor (e.g. gvim, emacs) to input content and \LaTeX\, commands in the file,
     e.g. sample.tex.

  \item Compile the file with the Unix command: {\bf latex sample.tex}
    
        The command produces a number of files:
        \begin{description}
        \item[sample.aux] This file contains information generated by \LaTeX\, on
        where to find things such as table of contents entries. Usually you
        can just ignore it.
        \item[sample.log] This file contains a detailed log of what \LaTeX\, did while
        processing your file. If you encounter errors that you just can't figure
        out, look in here for additional information which may help.
        \item[sample.dvi] This is the DeVice Independent (DVI) format output of
        what \LaTeX\, thinks your document should look like.
        \end{description}

  \item Display the sample.dvi file with the command: {\bf xdvi sample.dvi} or simply {\bf xdvi sample}

  \item Convert the dvi file to a postscript file with the command:
        {\bf dvips -o psfilename.ps sample.dvi}

  \item Print the sample.dvi file with the command:
        {\bf dviprint sample} or {\bf dvips sample.dvi \verb1|1 lpr -d sample.dvi}
 
\end{enumerate}


% \verb command simulates a short piece of typed text inside an ordinary
% paragraph. Its argument is enclosed by a pair of
% identical characters. The argument of the following \verb command is
% contained between the two 1 characters. Instead of 1, you can use any
% character that does not appear in the argument except a space, a letter,
% or a *.


\section{\LaTeX\, Document Structure}


% The verbatim environment allows you to type the text exactly the way you
% want it to appear in the document.

\begin{verbatim}
         \documentclass{article} 
           preamble 
         \begin{document} 
           body 
         \end{document} 
\end{verbatim}


\section{General Principles}
\begin{enumerate}
  \item
   % The symbols ` and ' create `Quotes' or ``Double quotes''
   All input, both text and formatting commands are in ``ASCII'' text.      
  \item
   Spaces and line breaks are not important. A blank line starts 
   a new paragraph, however. 

%Words are separated by one or more spaces. Paragraphs are separated by one
%or more blank lines. The output is not affected by adding extra spaces or
%extra blank lines to the input file. 
%Use a double backslash to break a line in a specific place.

  \item
   All commands start with a backslash:\\
   e.g. \verb1\documentclass1
  \item
   Braces are used for ``arguments'':\\
   e.g. \verb1\begin{document}1
  \item
   Square brackets are used for ``optional arguments'':\\
   e.g. \verb2\documentclass[11pt]{article}2
  \item
   Commands are case sensitive.\\
   e.g. \verb1\documentclass1 but not \verb1\DocumentClass1                                                
\end{enumerate}


\section{Special Symbols}
Some text characters must be generated by control sequences (i.e., quotes, \verb1{}, [], \1, etc.).  
The special symbols include
\verb1$ & % # _ { } ~ ^ \1.
Use a backslash in front of the symbols to correctly display the first seven symbols.
in the above list
To display the last three, use \verb1\verb1 command, e.g. \verb1\verb2\21.
The double backslash \verb1\\1 is used to start a new line.


\section{Fonts}
Since \LaTeX\, is a formatter, all changes in the format of text must be explicitly expressed.

 \begin{tabular}{|ll|} \hline
  What you type                                    &   What you see \\     
 \hline
  \verb1{\bf hello}1                               &    {\bf hello}    \\                                   
  \verb1{\em hello} is the same as {\it hello}1    &    {\em hello} is the same as {\it hello}\\
  \verb1{\underline {hello}}1                      &    {\underline {hello}} \\
 \hline
  \verb1{\tiny hello}1                             &    {\tiny hello} \\ 
  \verb1{\scriptsize hello}1                       &    {\scriptsize hello}\\ 
  \verb1{\footnotesize hello}1                     &    {\footnotesize hello}\\ 
  \verb1{\small hello}1                            &    {\small hello}\\ 
  \verb1{\normalsize hello}1                       &    {\normalsize hello}\\ 
  \verb1{\large hello}1                            &    {\large hello}\\ 
  \verb1{\Large hello}1                            &    {\Large hello}\\ 
  \verb1{\LARGE hello}1                            &    {\LARGE hello}\\ 
  \verb1{\huge hello}1                             &    {\huge hello}\\ 
  \verb1{\Huge hello}1                             &    {\Huge hello} \\  
 \hline 
 \end{tabular}


\section{Lists}
There are basically three types of lists in \LaTeX\,: 
\begin{itemize}
   \item    itemization (``bullets'') - \verb1\begin{itemize}1         
   \item    enumeration (1, 2, 3, ...) - \verb1\begin{enumerate}1
   \item    description - \verb1\begin{description}1 
\end{itemize}

All lists in \LaTeX\, have the same general format: 
\begin{verbatim}
       \begin{list-type} 
         \item list-entry 
         \item next-list-entry 
       \end{list-type} 
\end{verbatim}

\subsection{The Itemize Environment}
One can have lists within lists by using the \verb1\subitem1 or nested
itemize environments.

 \begin{verbatim}
  \begin{itemize}
    \item
     item one
     \subitem
      subitem one
     \subitem
      subitem two
   \item 
     item two
  \end{itemize}

  \begin{itemize}
    \item
     item one
     \begin{itemize}
      \item
      subitem 
    \end{itemize} 
    \item
     item two
  \end{itemize}
 \end{verbatim}

\subsection{The Enumerate Environment}
 \begin{verbatim}
  \begin{enumerate}
    \item
     item one
    \item
     item two
  \end{enumerate}
 \end{verbatim}

\subsection{The Description Environment}
Description lists are similar to enumerated or itemized lists. 
In a description list, each item has both a term and a description. 
 \begin{verbatim}
  \begin{description}
    \item[Step 1]
      Step 1 of the algorithm
    \item[Step 2]
      Step 2 of the algorithm
  \end{description}
 \end{verbatim}


\section{Typing Math in \LaTeX\,}
To type math expressions in the running text, use either of the two
short forms \verb1\(...)\1 or \verb1$...$1, 
e.g.  \verb1$\hat{\beta}$1.

One can also use the {\bf displaymath}, {\bf equation} and {\bf eqnarray} 
environments:

\begin{enumerate} 
 
  \item 
  {\bf displaymath}: \\\\
  e.g. 
  \begin{displaymath}
  {\rm logit}(p) = \log(\frac{p}{1-p})  \hspace{0.5in} \mbox{is the definition of logit of $p$}
  \end{displaymath}

  \item
  {\bf equation}: \\\\
  e.g.
  \begin{equation}
    \bar{X} = \frac{\sum_{i=1}^n X_i}{n}
  \end{equation}

  The \verb1equation*1 environment is the same as \verb1equation1 except it
  does not generate equation numbers. \\\\
  e.g. 
  \begin{equation*}
  Y_{ij} = \mu + \alpha_i + \beta_j + \epsilon_{ij}, \hspace{0.5in} \epsilon_{ij}\sim N(0,\sigma^2)
  \end{equation*}

  Big delimiters are most often used with array. \\\\
  e.g.
  \begin{equation}
   \left(\begin{array}{c}
   y\\
   z
   \end{array}\right)
  \end{equation}
 
  The \verb1\left1 and \verb1\right1 commands must come in matching pairs, but
  the matching delimiters need not be the same. \\\\
  e.g. 
  \begin{equation}
  \left.\frac{\partial f(x, y)}{\partial x}\right|_{x=x_0}=y
  \end{equation}

  We can type text in math mode by using the \verb1\mbox1 command. \\\\
  e.g.
  \begin{equation}
   x=\left\{ \begin{array}{ll}
   y& \mbox{if $y>0$}\\
   z+y & \mbox{otherwise}
   \end{array}\right.
  \end{equation}

  \item
  {\bf eqnarray}: \\\\ 
  The \verb1displaymath1 and \verb1equation1 environments make one-line
  formulas. A sequence of equations or inequalities is displayed with the 
  \verb1eqnarray1 environment. By default, an equation number is put in
  every single line. Use the \verb1\nonumber1 command in each of the lines
  you want to suppress an equation number. \\\\
  e.g.
  \begin{eqnarray}
   x&\ll &y_{1}+\cdots +y_{n}\\
   &\leq &z \nonumber
  \end{eqnarray}

  The \verb1eqnarray*1 environment is the same as the \verb1eqnarray1 
  environment except it does not generate equation numbers. \\\\
  e.g.
  \begin{eqnarray*}
   &A\stackrel{a'}{\to}B\stackrel{b'}{\to}C\\
   &\vec{x}\stackrel{\rm def}{=} (x_1,\ldots, x_n)
  \end{eqnarray*}

  For multiline formula, sometimes it is desirable to have one equation number
  and put the equation number in the middle line of the equations. In this
  case, we can combine the \verb1equation1 environment and the \verb1split1
  environment. \\\\
  e.g.
  \begin{equation}
   \begin{split}
    H_0 : \mu_1 &= \mu_2 \\ 
    H_1: \mu_1 &\neq \mu_2
   \end{split}
  \end{equation}

  We cannot use the \verb1\bf1 command in math mode. Instead we use
  \verb1\mathbf1 for English characters and \verb1\boldsymbol1 for both English
  characters and Greek letters. \\\\
  e.g.
  \begin{equation}
    \Biggl\{
      \begin{array}{c}
       {\mathbf Y_i} = {\mathbf X_i} \boldsymbol{\beta} + {\mathbf Z_i} \boldsymbol{b_i} + \boldsymbol{e_i} \\               
       \boldsymbol{b_i} \sim {\rm N}(0,{\mathbf D}), \boldsymbol{e_i} \sim {\rm N}(0,{\mathbf R_i}),
             \boldsymbol{b_i} \perp \boldsymbol{e_i}
      \end{array}
  \end{equation}                 

\end{enumerate}

\section{Tables and Figures}
To create tables, you will need the {\bf tabular} environment.

  \begin{tabular}{|c||c|c|c|c|}\hline
    Source of Variation & SS  & df & MS  & F-ratio \\
  \hline
    Treatment           & SST & 2  & MST & $\frac{MST}{MSE}$ \\
    Error               & SSE & 30 & MSE &  \\
  \hline
  \end{tabular}

%r - right justify the column 
%l - left justify the column 
%c - center the column 

To insert graphics into \LaTeX, use the {\bf figure} environment. 

  \begin{figure}
    \includegraphics[width=6.5in, height=7.5in]{hist.eps}
  \end{figure}

\section{Centering and ``Flushing"}
There are several \LaTeX\, environments to control 
the alignment of text.
\begin{verbatim}
     Center: \begin{center} ... \end{center}
     Right Align: \begin{flushright} ... \end{flushright}
     Left Align: \begin{flushleft} ... \end{flushleft}
\end{verbatim}

The following declarations can be used at the beginning of the body to 
produce alignment effect for the entire document. 
Be sure to put the declaration immediately after the 
\verb1\begin{document}1 command. 
These declarations will be turned off as soon as they 
encounter a \verb1\end{...}1 command. 

\begin{verbatim}
     \centering
     \raggedleft
     \raggedright
\end{verbatim}


\section{Errors in Running \LaTeX\,}

When you compile a \LaTeX\, file that contains syntax errors, \LaTeX\, will
print out error messages, indicate which line contains an error,  
and print a ``?'' prompt. You can either enter letter ``x'' to exit
(quit the compilation) or enter ``E'' to edit the \LaTeX\, file.

%  $x+y


\end{document}
