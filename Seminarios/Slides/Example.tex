\documentclass[pdf]{prosper}

% Example for the `HA-Prosper' package.
% Created by: Hendri Adriaens
%             http://center.uvt.nl/phd_stud/adriaens
%             Center for Economic Research
%             Tilburg University, the Netherlands

%================================================
% Please also read the manual of HA-prosper and
% of the specific style you are using since some
% features of this example might not be supported
% by the style you use.
%================================================

\usepackage[toc,highlight,HA,notes,portrait,hlsections]{HA-prosper}

\title{Example for the HA-prosper package}
\subtitle{A package for use with prosper}
\author{Hendri Adriaens\\
\institution{CentER}\\
\institution{\href{http://center.uvt.nl/phd_stud/adriaens}{http://center.uvt.nl/phd\string_stud/adriaens}}}

\DefaultTransition{Wipe}
\TitleSlideNav{FullScreen}
\NormalSlideNav{ShowBookmarks}
\LeftFoot{\href{http://center.uvt.nl/phd_stud/adriaens}{Hendri Adriaens}, \today}
\RightFoot{Example for the HA-prosper package}


\begin{document}

% ==================================================================================
% Slide 1
\maketitle
% ==================================================================================


% ==================================================================================
% Slide 2
\begin{slide}{Introduction}
\begin{itemize}
\item Welcome to the example for the HA-prosper package. \item
This example demonstrates some of the possibilities of HA-prosper.
\item See the style-specific examples for a demonstration of
features implemented by a style.
\end{itemize}
\end{slide}
% ==================================================================================


% No slide
\tsection{Section 1}


% ==================================================================================
% Slide 3
\begin{slide}{Section command}
\begin{itemize}
\item The section in the table of contents called `Section 1' has been
created by the `$\backslash$section' command;
\item This command does not generate a slide.
\end{itemize}
\end{slide}
% ==================================================================================


% Slide 4
\part[toc=Part 1!!!,template=wideslide,trans=Dissolve]{Part 1}


% ==================================================================================
% Slide 5
\begin{slide}{Part command}
\begin{itemize}
\item The previous slide has been
created by the `$\backslash$part' command;
\item This command does generates a slide but not a table of contents section;
\item This command has the following optional arguments:
\begin{itemize}
\item toc: overwrites toc entry;
\item trans: specific transition effect;
\item template: slide template that should be used to create the slide.
This template should be supported as a regular slide template by the
style that you use for your presentation.
\end{itemize}
\item Regular slides also support the first two options.
\end{itemize}
\end{slide}
% ==================================================================================


% Slide 6
\tsectionandpart[toc=Section 2!!!]{Section 2}


% ==================================================================================
% Slide 7
\begin{slide}{Sectionandpart command}
\begin{itemize}
\item The section in the table of contents called `Section 2' has been
created by the `$\backslash$sectionandpart' command;
\item This command generates a slide and a table of contents section;
\item This command has the same optional arguments as the `$\backslash$part'
command.
\end{itemize}
\end{slide}
% ==================================================================================


% ==================================================================================
% Slide 8
\overlays{2}{
\begin{slide}{Numbering and overlays}
\begin{itemstep}
\item This overlay contains an equation:
\begin{equation}
\label{eq:1}
(a+b)^n=\sum_{k=0}^n\left(\begin{array}{l}n\\k\end{array}\right)a^{n-k}b^k
\end{equation}
\item It is equation number~\ref{eq:1}.
\end{itemstep}
\end{slide}
}
% ==================================================================================


% ==================================================================================
% Slide 9
\begin{slide}[toc=]{Notes}
\begin{itemize}
\item This slide contains notes.
\item HA-prosper package options for notes:
\begin{itemize}
\item slidesonly (default): produces only slides. Recommended for the
actual presentation.
\item notesonly: produces only the notes.
\item notes: includes the notes into the presentation for printing
purposes or for distributing the entire presentation.
\end{itemize}
\item If you want to print only the notes, first run the presentation
with the option `notes' to create page number labels, then run it with
the option `notesonly'.
\item (Notice that this slide does not have a table of contents entry.)
\end{itemize}
\end{slide}
% ==================================================================================


% ==================================================================================
% Notes for slide 9
\begin{notes}{Notes for slide 9}
My notes for slide 9.
\end{notes}
% ==================================================================================


% ==================================================================================
% Slide 10
\overlays{6}{
\begin{slide}{Itemstep type 0 example}
The following slides contain examples of the different `itemstep'
environments.
\begin{itemstep}[stype=0,sstart=2]
\item item 1
\begin{itemstep}
\item item 1.1
\begin{itemstep}
\item item 1.1.1
\end{itemstep}
\item item 1.2
\end{itemstep}
\item item 2
\end{itemstep}
\end{slide}
}
% ==================================================================================


% ==================================================================================
% Slide 11
\overlays{5}{
\begin{slide}{Itemstep type 1 example}
\begin{itemstep}[stype=1]
\item item 1
\begin{itemstep}
\item item 1.1
\begin{itemstep}
\item item 1.1.1
\end{itemstep}
\item item 1.2
\end{itemstep}
\item item 2
\end{itemstep}
\end{slide}
}
% ==================================================================================


% ==================================================================================
% Slide 12
\overlays{5}{
\begin{slide}{Itemstep type 2 example}
\begin{itemstep}[stype=2]
\item item 1
\begin{itemstep}
\item item 1.1
\begin{itemstep}
\item item 1.1.1
\end{itemstep}
\item item 1.2
\end{itemstep}
\item item 2
\end{itemstep}
\end{slide}
}
% ==================================================================================

\end{document}
