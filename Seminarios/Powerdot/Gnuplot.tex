\begin{slide}[toc=,bm=]{Sumário}
\tableofcontents[content=all] %Outra opção: content=sections
\end{slide}

\begin{slide}{Sumário}
  \begin{enumerate}
    \item Funções Básicas\label{s1} \pause
    \item Gráfico de Funções \pause
    \begin{itemize}[type=1]
       \item Gráficos unidimensionais 
       \item Gráficos bidimensionais 
       \item Gráficos de Funções Paramétricas
       \item Gráficos em coordenadas polares \pause
    \end{itemize}
    \item Gráficos de Arquivos \pause[2] % Uma pausa mais longa
    \begin{itemize}[type=2]
       \item Gráficos unidimensionais 
       \item Gráficos bidimensionais \pause
    \end{itemize}
    \item Tipos de Terminais \pause
    \item Scripts.
  \end{enumerate}
\end{slide}

\section[tocsection=true]{Funções Básicas}

\begin{slide}[toc=Funções Básicas]{Funções Básicas I}
 
\begin{description}
\item[$\abs(x)$] Para $x$ qualquer, retorna o valor absoluto de $x$ com o mesmo tipo de $x$.
\item[$\abs(x)$] Para $x$ qualquer, retorna $\sqrt{\Re(x)^2+\Im(x)^2}$ para $x$ complexo.
\item[$\acos(x)$] Para $x$ qualquer, retorna um valor entre $-\pi \leq acos(x) \leq \pi$ se $-1\leq x \leq 1$.
\item[$\acosh(x)$] Para $x$ qualquer, retorna um valor entre $-\pi \leq acos(x) \leq \pi$ se $ x \geq 1$.
\item[$arg(x)$] Para $x$ qualquer, retorna a fase de $x$.
\item[$asin(x)$] Para $x$ qualquer, retorna o $\sen^{-1} x$ (inverso do seno).
\item[$asinh(x)$] Para $x$ qualquer, retorna o $\senh^{-1} x$ (inverso do seno hiperbólico).
\end{description}
\end{slide}

\begin{slide}[toc=]{Funções Básicas II}
 
\begin{description}
\item[$atan(x)$] Para $x$ qualquer, retorna o $\tg^{-1} x$ (inverso da tangente).
\item[$atan2(y,x)$] Para $x$ e $y$ inteiros ou reais, retorna  $\tg^{-1}(y/x)$ (inverso da tangente).
\item[$atanh(x)$] Para $x$ qualquer, retorna o $\tgh^{-1} x$ (inverso da tangente hiperbólica).
\item[$besj0(x)$] Para $x$ inteiro ou real, retorna a função de Bessel $j_{0}(x)$.
\item[$besj1(x)$] Para $x$ inteiro ou real, retorna a função de Bessel $j_{1}(x)$.
\item[$besy0(x)$] Para $x$ inteiro ou real, retorna a função de Bessel $y_{0}(x)$.
\item[$besy1(x)$] Para $x$ inteiro ou real, retorna a função de Bessel $y_{1}(x)$.
\end{description}
\end{slide}

\begin{slide}[toc=]{Funções Básicas III}
 
\begin{description}
\item[$ceil(x)$] Para $\lceil x \rceil$, retorna o menor inteiro não menor que $x$ (parte real).
\item[$cos(x)$]  Para $x$ qualquer, retorna o cosseno de $x$, $\cos x$.
\item[$cosh(x)$] Para $x$ qualquer, retorna o cosseno hiperbólico de $x$, $\cosh x$, .
\item[$erf(x)$] Para $x$ qualquer, retorna a função erro $\hbox{Erf}(\hbox{real}(x))$, do real($x$).
\item[$erfc(x)$] Para $x$ qualquer, retorna $\hbox{Erfc}(\hbox{real}(x)) = 1.0 - \hbox{Erf}(\hbox{real}(x))$.
\item[$exp(x)$] Para $x$ qualquer, retorna $e^{x}$.
\item[$floor(x)$] Para $x$ qualquer, retorna $\lfloor x \rfloor$, o maior inteiro não maior do que $x$ (parte real).
\end{description}
\end{slide}
\begin{slide}[toc=]{Este título não aparece no índice}
Um pequeno exemplo

Note que o título do slide não aparece no menu lateral do slide
\end{slide}

%==============================================================
\begin{slide}[toc=]{Um slide qualquer}
dsakfjçsdak
dsakfjk
dsafas


dsafsdaf

no slide \ref{s1} viu-se que


\end{slide}


%==============================================================
\begin{slide}{Usando o tcolorbox}
\begin{tcolorbox}
This is another \textbf{tcolorbox}.
\tcblower
Here, you see the lower part of the box.
\end{tcolorbox}

\begin{tcolorbox}[colback=red!5!white,colframe=red!75!black,title=Cabeçalho da caixa]
This is another \textbf{tcolorbox}.
\tcblower
Here, you see the lower part of the box.
\end{tcolorbox}

\end{slide}

% \tcbset{colback=white,arc=0mm,width=(\linewidth-4pt)/4,
% equal height group=AT,before=,after=\hfill,fonttitle=\bfseries}

%==============================================================
\begin{slide}[toc=]{}
\begin{tcolorbox}[title=tese,colframe=red!75!black]
Some content.
\end{tcolorbox}

\end{slide}

%==============================================================
\begin{slide}[toc=]{}
\begin{tcolorbox}[sidebyside,title=Lower separated]
This is the upper part.

\tcblower

This is the lower part.
\end{tcolorbox}
\end{slide}

\tcbset{colback=white,arc=0mm,width=(0.6\linewidth),
 equal height group=AT,before=,after=\hfill,fonttitle=\bfseries}


%==============================================================
\begin{slide}[toc=]{Mudando o tamanho da caixa}
\begin{flushright}
\begin{tcolorbox}[sidebyside,title=Lado a lado]
Caixa de tamnho 60\% da largura da linha 

\tcblower

Caixa deslocada para direita
\end{tcolorbox}  
\end{flushright}

\end{slide}

\begin{slide}{Simple onslide+}
\texttt{onslide }: \onslide{1}{power}\onslide{2}{dot}\\
\texttt{onslide+}: \onslide+{1}{power}\onslide+{2}{dot}
\end{slide}
