\begin{slide}{Sum�rio}
  \begin{enumerate}
    \item Fun��es B�sicas\label{s1} \pause
    \item Gr�fico de Fun��es \pause
    \begin{itemize}
       \item Gr�ficos unidimensionais 
       \item Gr�ficos bidimensionais 
       \item Gr�ficos de Fun��es Param�tricas
       \item Gr�ficos em coordenadas polares \pause
    \end{itemize}
    \item Gr�ficos de Arquivos \pause
    \begin{itemize}
       \item Gr�ficos unidimensionais 
       \item Gr�ficos bidimensionais \pause
    \end{itemize}
    \item Tipos de Terminais \pause
    \item Scripts.
  \end{enumerate}
\end{slide}

\section{Fun��es B�sicas}

\begin{slide}{Fun��es B�sicas I}
 
\begin{description}
\item[$\abs(x)$] Para $x$ qualquer, retorna o valor absoluto de $x$ com o mesmo tipo de $x$.
\item[$\abs(x)$] Para $x$ qualquer, retorna $\sqrt{\Re(x)^2+\Im(x)^2}$ para $x$ complexo.
\item[$\acos(x)$] Para $x$ qualquer, retorna um valor entre $-\pi \leq acos(x) \leq \pi$ se $-1\leq x \leq 1$.
\item[$\acosh(x)$] Para $x$ qualquer, retorna um valor entre $-\pi \leq acos(x) \leq \pi$ se $ x \geq 1$.
\item[$arg(x)$] Para $x$ qualquer, retorna a fase de $x$.
\item[$asin(x)$] Para $x$ qualquer, retorna o $\sen^{-1} x$ (inverso do seno).
\item[$asinh(x)$] Para $x$ qualquer, retorna o $\senh^{-1} x$ (inverso do seno hiperb�lico).
\end{description}
\end{slide}

\begin{slide}{Fun��es B�sicas II}
 
\begin{description}
\item[$atan(x)$] Para $x$ qualquer, retorna o $\tg^{-1} x$ (inverso da tangente).
\item[$atan2(y,x)$] Para $x$ e $y$ inteiros ou reais, retorna  $\tg^{-1}(y/x)$ (inverso da tangente).
\item[$atanh(x)$] Para $x$ qualquer, retorna o $\tgh^{-1} x$ (inverso da tangente hiperb�lica).
\item[$besj0(x)$] Para $x$ inteiro ou real, retorna a fun��o de Bessel $j_{0}(x)$.
\item[$besj1(x)$] Para $x$ inteiro ou real, retorna a fun��o de Bessel $j_{1}(x)$.
\item[$besy0(x)$] Para $x$ inteiro ou real, retorna a fun��o de Bessel $y_{0}(x)$.
\item[$besy1(x)$] Para $x$ inteiro ou real, retorna a fun��o de Bessel $y_{1}(x)$.
\end{description}
\end{slide}

\begin{slide}{Fun��es B�sicas III}
 
\begin{description}
\item[$ceil(x)$] Para $\lceil x \rceil$, retorna o menor inteiro n�o menor que $x$ (parte real).
\item[$cos(x)$]  Para $x$ qualquer, retorna o cosseno de $x$, $\cos x$.
\item[$cosh(x)$] Para $x$ qualquer, retorna o cosseno hiperb�lico de $x$, $\cosh x$, .
\item[$erf(x)$] Para $x$ qualquer, retorna a fun��o erro $\hbox{Erf}(\hbox{real}(x))$, do real($x$).
\item[$erfc(x)$] Para $x$ qualquer, retorna $\hbox{Erfc}(\hbox{real}(x)) = 1.0 - \hbox{Erf}(\hbox{real}(x))$.
\item[$exp(x)$] Para $x$ qualquer, retorna $e^{x}$.
\item[$floor(x)$] Para $x$ qualquer, retorna $\lfloor x \rfloor$, o maior inteiro n�o maior do que $x$ (parte real).
\end{description}
\end{slide}
\begin{slide}[toc=]{Este t�tulo n�o aparece no �ndice}
Um pequeno exemplo

Note que o t�tulo do slide n�o aparece no menu lateral do slide
\end{slide}

%==============================================================
\begin{slide}[toc=]{}
dsakfj�sdak
dsakfjk
dsafas


dsafsdaf

no slide \ref{s1} viu-se que


\end{slide}

