\documentclass[blue]{beamer}
\usepackage{beamerthemesplit}
\setbeamercolor{alerted text}{fg=blue}
\logo{\includegraphics[height=1cm,width=1cm]{iitblogo.pdf}}
\begin{document}
\title[\hspace{1cm}Spoken tutorial: What is compilation?
\hspace{0.5cm}\insertframenumber/\inserttotalframenumber]
{Spoken tutorial: What is compilation?}
\author[Kannan Moudgalya] {Kannan M. Moudgalya}
\begin{frame}
\frametitle{Benefits of \LaTeX}
\end{frame}



\begin{frame}
\frametitle{\LaTeX}
\begin{itemize}
\item<+-|alert@+> \LaTeX\ is an excellent typesetting software.
\item<+-|alert@+> Quality of documents produced by \LaTeX\ is unmatched.
\item<+-|alert@+> \LaTeX\ is free and open source.
\item<+-|alert@+> It is available on windows and all unix systems, including, Mac and linux.
\item<+-|alert@+> \LaTeX\ has outstanding features, such as, automatic numbering of equations, chapters and sections, figures, and tables.
\item<+-|alert@+> Documents with a lot of mathematical equations
  can also be generated easily in \LaTeX.
\item<+-|alert@+> It is easy to produce bibliographic entries,
  with changeable format, on the fly.
\item<+-|alert@+> With \LaTeX\ taking care of formating, the writer can concentrate on more important activities, such as, content generation and logical sequencing of ideas.
\end{itemize}
\end{frame}

\begin{frame}
\frametitle{More Spoken Tutorials on \LaTeX}
\end{frame}

\begin{frame}
\frametitle{Spoken Tutorials on \LaTeX}
Visit {\color{magenta}www.moudgalya.org} for following spoken
tutorials on \LaTeX
%The following spoken tutorials on \LaTeX\ are available at {\color{magenta}www.moudgalya.org}:
\begin{enumerate}
\item<+-|alert@+> What is compilation?
\item<+-|alert@+> Letter writing
\item<+-|alert@+> Report writing
\item<+-|alert@+> Mathematical typesetting
\item<+-|alert@+> Equations
\item<+-|alert@+> Tables and figures
\item<+-|alert@+> How to create bibliography?
\item<+-|alert@+> Inside story of bibliography
\end{enumerate}
\begin{itemize}
\item<+-|alert@+> This order of tutorials is recommended for best results
\item<+-|alert@+> Source files used to create these tutorials
  also are available at this web site 
\item<+-|alert@+> Plan to add a tutorial on 
  installation of \LaTeX\ in windows OS
\item<+-|alert@+> There will also be other tutorials in the near
  future, for example, beamer for slide presentation
\item<+-|alert@+> This presentation has been made with beamer,
  using \LaTeX\
\end{itemize}
\end{frame}

\begin{frame}
\frametitle{Closing Tips}
\begin{itemize}
\item<+-|alert@+> Go through as many of the spoken tutorials, as
  possible
\item<+-|alert@+> Practise them in parallel
\item<+-|alert@+> Start with a working \LaTeX\ file
\item<+-|alert@+> Make one change at a time, save, compile and
  make sure that what you have done works, before making further
  changes 
\item<+-|alert@+> Remember to save the source file before
  compiling 
\end{itemize}
\end{frame}

\begin{frame}
\frametitle{Additional Help}
\begin{itemize}
\item<+-|alert@+> There are many books on \LaTeX.  Recommend two: 
\item<+-|alert@+> Book from the original creator of \LaTeX,
  Leslie Lamport.  {\color{magenta} \LaTeX: A document
    preparation system: User's guide and reference manual.}
  Addison Wesley, Reading, MA, USA, 2nd edition, 1994.
\item<+-|alert@+> This book is available in a low cost Indian
  edition as well.
\item<+-|alert@+> Advanced users may consult the following book:
\item<+-|alert@+> F. Mittelbach, M. Goossens and others.
  {\color{magenta} The \LaTeX\ companion.} Addison
    Wesley, Reading, MA, USA, 2nd edition, 2004.
\item<+-|alert@+> The first book and a web search is usually
  sufficient for most purposes, however.
\item<+-|alert@+> The main site for all \LaTeX\ related material
  is {\color{magenta}www.ctan.org.}
\end{itemize}
\end{frame}

\begin{frame}
\frametitle{Project Support}
\begin{itemize}
\item<+-|alert@+> Funding for this work has come from the
  {\color{magenta} National Mission on Education through ICT},
  launched by the Ministry of Human Resources Development,
  Government of India.
\item<+-|alert@+> URL for this mission: {\color{magenta}
    www.sakshat.ac.in}.
\item<+-|alert@+> Spoken tutorial activity is the initiative of
  the \emph{Talk to a Teacher} project of the mission,
  coordinated by CDEEP, IIT Bombay: {\color{magenta}
    www.cdeep.iitb.ac.in}. 
\item<+-|alert@+> Use of spoken tutorial
  to popularise software development and its use will be
  coordinated through {\color{magenta} www.fossee.in}.
\item<+-|alert@+> \emph{Fossee} stands for Free and Open source
  Software in Science and Engineering Education.  This project
  also is supported by the national mission on education.
\item<+-|alert@+> Keep watching these links for more spoken
  tutorials and their translation in other languages.
\end{itemize}
\end{frame}

\begin{frame}
\frametitle{Thank you}
\begin{itemize}
\item Please send your feedback to kannan@iitb.ac.in
\end{itemize}
\end{frame}

\end{document}

\item<+-|alert@+> 
\item<+-|alert@+> 
