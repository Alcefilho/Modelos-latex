\section{\texorpdfstring{\href{https://tex.stackexchange.com/questions/102384/how-to-have-each-slide-in-a-separate-file-each-compilable-then-combine-them}{How
to have each slide in a separate file, each compilable, then combine
them}}{How to have each slide in a separate file, each compilable, then combine them}}\label{how-to-have-each-slide-in-a-separate-file-each-compilable-then-combine-them}

https://tex.stackexchange.com/questions/102384/how-to-have-each-slide-in-a-separate-file-each-compilable-then-combine-them

I would like to write each slide of a presentation into its own file, so
that I can compile and view each slide, one at a time.\\
But, I still need to combine all of them to make my presentation.\\
What's is the way to do that?\\
I try the following, but it doesn't compile.\\
\emph{includedoc} doesn't seems to be there in Beamer.

file: 1.tex

\emph{\textbackslash{}documentclass\{beamer\}}

\emph{\textbackslash{}begin\{document\}}

\emph{ \textbackslash{}begin\{frame\}}

\emph{ Content}

\emph{ \textbackslash{}end\{frame\}}

\emph{\textbackslash{}end\{document\}}

file: slide.tex

\emph{\textbackslash{}documentclass\{beamer\}}

\emph{\textbackslash{}begin\{document\}}

\emph{ \textbackslash{}includedoc\{1\}}

\emph{\textbackslash{}end\{document\}}

\subsection{Method 1}\label{method-1}

\begin{quote}
The structure of LATEX documents doesn't allow one standalone document
to be included as a part of another. \emph{docmute}package tries to
remedy this limitation and enables standalone documents to be imported
with the standard \emph{\textbackslash{}input} and
\emph{\textbackslash{}include} commands just as if they had no preamble.
\end{quote}

Use \emph{docmute} package in the main input file. \emph{docmute}
package will import everything between
\emph{\textbackslash{}begin\{document\}} and
\emph{\textbackslash{}end\{document\}} of the imported file
(\emph{child.tex} in the following example). You can still compile the
\emph{child.tex} as a standalone presentation.

\emph{\textbackslash{}documentclass\{beamer\}}

\emph{\textbackslash{}usepackage\{filecontents\}}

\emph{\textbackslash{}begin\{filecontents*\}\{child.tex\}}

\emph{\textbackslash{}documentclass\{beamer\}}

\emph{\textbackslash{}begin\{document\}}

\emph{ \textbackslash{}begin\{frame\}}

\emph{ Content}

\emph{ \textbackslash{}end\{frame\}}

\emph{\textbackslash{}end\{document\}}

\emph{\textbackslash{}end\{filecontents*\}}

\emph{\textbackslash{}usepackage\{docmute\}}

\emph{\textbackslash{}begin\{document\}}

\emph{ \textbackslash{}input\{child\}}

\emph{\textbackslash{}end\{document\}}

\subsection{Method 2}\label{method-2}

Using \emph{standalone} document class with option \emph{beamer} for the
child (imported) input file and using package \emph{standalone} in the
main input file.

\emph{\textbackslash{}documentclass\{beamer\}}

\emph{\textbackslash{}usepackage\{filecontents\}}

\emph{\textbackslash{}begin\{filecontents*\}\{child.tex\}}

\emph{\textbackslash{}documentclass{[}beamer{]}\{standalone\}}

\emph{\textbackslash{}begin\{document\}}

\emph{ \textbackslash{}begin\{frame\}}

\emph{ Content}

\emph{ \textbackslash{}end\{frame\}}

\emph{\textbackslash{}end\{document\}}

\emph{\textbackslash{}end\{filecontents*\}}

\emph{\textbackslash{}usepackage\{standalone\}}

\emph{\textbackslash{}begin\{document\}}

\emph{ \textbackslash{}input\{child\}}

\emph{\textbackslash{}end\{document\}}

Note: using the first method seems to be simpler!

One option, but I am not sure if this is \emph{exactly} what you are
after, would be to label the frames so that you can use
\emph{\textbackslash{}includeonlyframes}; a little example:

\emph{\textbackslash{}documentclass\{beamer\}}

\emph{\textbackslash{}includeonlyframes\{one,three\}}

\emph{\textbackslash{}begin\{document\}}

\emph{\textbackslash{}begin\{frame\}{[}label=one{]}}

\emph{test frame one.}

\emph{\textbackslash{}end\{frame\}}

\emph{\textbackslash{}begin\{frame\}{[}label=two{]}}

\emph{test frame two.}

\emph{\textbackslash{}end\{frame\}}

\emph{\textbackslash{}begin\{frame\}{[}label=three{]}}

\emph{test frame three.}

\emph{\textbackslash{}end\{frame\}}

\emph{\textbackslash{}begin\{frame\}{[}label=four{]}}

\emph{test frame four.}

\emph{\textbackslash{}end\{frame\}}

\emph{\textbackslash{}end\{document\}}

Using the code above, only the two frames corresponding to the labels
\emph{one} and \emph{three} will be shown.

Combining this idea with the use of \emph{\textbackslash{}include} (or
\emph{\textbackslash{}input}), you then might write each frame contents
in its own file; something like

\emph{\textbackslash{}begin\{frame\}{[}label=two{]}}

\emph{the frame contents...}

\emph{\textbackslash{}end\{frame\}}

(notice, in particular that this subsidiary files have no
\emph{\textbackslash{}documentclass}, nor
\emph{\textbackslash{}begin\{document\}},
\emph{\textbackslash{}end\{document\}} commands); each of this files
will then be saved in your current working directory as, for example,
\emph{frameone.tex}, \emph{frametwo.tex}, \emph{framethree.tex},
\emph{framefour.tex} and then you can \emph{\textbackslash{}include}
them in the desired order in your "main" document; for example:

\emph{\textbackslash{}documentclass\{beamer\}}

\emph{\textbackslash{}begin\{document\}}

\emph{\textbackslash{}include\{frametwo\}}

\emph{\textbackslash{}include\{framefour\}}

\emph{\textbackslash{}include\{framethree\}}

\emph{\textbackslash{}include\{frameone\}}

\emph{\textbackslash{}end\{document\}}

will show the four frames in the order \emph{two}, \emph{four},
\emph{three}, \emph{one}.

The following code

\emph{\textbackslash{}documentclass\{beamer\}}

\emph{\textbackslash{}includeonlyframes\{one,three\}}

\emph{\textbackslash{}begin\{document\}}

\emph{\textbackslash{}include\{frametwo\}}

\emph{\textbackslash{}include\{framefour\}}

\emph{\textbackslash{}include\{framethree\}}

\emph{\textbackslash{}include\{frameone\}}

\emph{\textbackslash{}end\{document\}}

will give you a two frame document showing the frames in the order
\emph{three}, \emph{one}.
