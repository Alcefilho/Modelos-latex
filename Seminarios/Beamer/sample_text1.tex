\documentclass[12pt]{article}
\usepackage{flexiprogram}
\usepackage{pstricks}
\usepackage{pst-node}

\bibliographystyle{plain}

\title{My First Document}
\author{I \and Me \and Myself}
\date{1 Jan 1000}

\renewcommand{\baselinestretch}{1.3}

\begin{document}
\maketitle

\begin{abstract}
\footnotesize
\sffamily

Today i.e. on \today   we learnt a little bit of \LaTeX.
Today i.e. on \today   we learnt a little bit of \LaTeX.
Today i.e. on \today   we learnt a little bit of \LaTeX.
Today i.e. on \today   we learnt a little bit of \LaTeX.
Today i.e. on \today   we learnt a little bit of \LaTeX.

\normalsize
Today i.e. on \today   we learnt a little bit of \LaTeX.
Today i.e. on \today   we learnt a little bit of \LaTeX.
Today i.e. on \today   we learnt a little bit of \LaTeX.
\end{abstract}

\section{Introduction}

\begin{itemize}
\item This is the first item.
\item And this is the second.
\end{itemize}

This is a run in formula defining $f(x_i) = \prod_{k=0}^{k=l} x^{i+1}_{i-1} +
5$.

Here is its displayed version
\[f(x_i) = \prod_{k=0}^{k=l} x^{i+1}_{i-1} + 5\]

\subsection{one} 

\subsubsection{Isn't it funny?}

As shown in \cite{Nie83indiaH} it
\nocite{MyNi83icalp}


\begin{eqnarray}
A & = & \{ a, b, c, \} \label{eq:1} \\ \nonumber
B & = & \left[ \begin{array}{c}\emptyset \\ \emptyset \end{array}\right]
\end{eqnarray}

Equation \ref{eq:1} shows something which I don't know.

\begin{table}[h]
\begin{center}
\begin{tabular}{|l|c|c|r|}
\hline
one & two & three & four \\ \hline
1 & 2 & 
\begin{tabular}{|l|l|}
a & this is a citation\cite{MyNi83icalp} \\ \hline
c & d \\ \hline
\end{tabular} & 4 \\ \hline
\end{tabular}
\caption{This is my first figure}
\end{center}
\end{table}

\begin{pspicture}(0,0)(10,10)
\rput(5,6){\rnode{n1}{\psframebox{This is a box}}}
\rput(10,2){\rnode{n2}{\pscirclebox{This is another box}}}
\ncline[doubleline=true,linewidth=.5mm,nodesepB=-.3cm]{->}{n1}{n2}
\end{pspicture}




\begin{program}{10}
\FL \ \PROCEDURE {\em fun\/}$(i,j)$
\NL{0} \OB this is the first line
\NL{1}  this line is indented further
\NL{1}   \IF (some condition)
\NL{1}	 \OB  do this
\NL{1}   \CB
\NL{1}   \ELSE
\NL{1}   \OB
\NL{1}   \FLCOMMENT{this }
\NL{2}      do that
\NL{1}   \CB \COMMENT{comment}
\NL{0} \CB
\end{program}










\appendix
\section{This is a new section}

\renewcommand{\thesection}{\arabic{section}}

\section{This is another section}


\bibliography{sample}

\end{document}

