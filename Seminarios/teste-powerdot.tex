\documentclass[style=sailor,clock,mode=present]{powerdot}
\usepackage[brazil]{babel}
\usepackage[utf8]{inputenc}
\usepackage{lipsum}
\usepackage{hyperref}

\pdsetup{
  palette=River,
  lf=\textcolor{red}{XXIII Semana da Física},
  rf=\textcolor{green}{Salviano A. Leão},
  trans=Box
}

\title{Meu primeiro slide LaTeX}
\author{Salviano A. Leão}
\date{\today}

\begin{document}
\maketitle

\section{Introdução}

\begin{slide}{Título do slide}
\lipsum[1]
\end{slide}



%==============================================================
\begin{slide}[toc=]{Segundo slide}
\begin{itemize}
   \item \url{https://sigaa.sistemas.ufg.br/}

   \item Link escondido \href{https://www.uol.com.br/}{UOL}
\end{itemize}

\end{slide}

%==============================================================
\begin{slide}{Equações}
O que ocorreu na linha acima? Não entendi, porém 
eq. de Maxwell \ref{eq:Max1} fornece:

\begin{equation}\label{eq:2LN} 
 \vec{F} = m \vec{a}
\end{equation} 


\begin{equation}\label{eq:Sch} 
 i \hbar \frac{\partial \psi}{\partial t} = H \psi
\end{equation} 


\begin{equation}\label{eq:Max1} 
 \nabla \times \vec{E} = - \frac{\partial \vec{B}}{\partial t}
\end{equation} 

\end{slide}



\end{document}
