\documentclass[a4paper,12pt,final,notitlepage]{article}

\usepackage{ucs}
\usepackage[utf8]{inputenc}
\usepackage{amsmath}
\usepackage{amsfonts}
\usepackage{amssymb}
\usepackage{amsthm}
\usepackage{multicol}
\usepackage{rotating}
\usepackage{xcolor}
\usepackage[brazil,brazilian]{babel}
\usepackage[T1]{fontenc}
\usepackage{graphicx}

\usepackage{hyperref}
\hypersetup{%
   bookmarks=true,%
   bookmarksnumbered=,%
   pdfauthor=Salviano A. Le\~{a}o,%
   pdfcreator=Salviano A. Le\~{a}o,%
   pdfproducer=Salviano A. Leão,%
   pdftitle=Titulo do trabalho%
}

\author{Rajarshi Guha}
\title{\href{http://freecode.com/articles/making-presentations-with-latex-and-prosper}%
       {Making Presentations with LaTeX and Prosper}}
\date{12-09-2013}

\begin{document}



\section{Introduction}

  A number of dedicated presentation programs have been written for Unix
systems, but they may not serve your needs if you have special
requirements, especially the need to display mathematical formulas.
The Prosper package can help you create attractive presentations while
letting you use the full power of LaTeX.



If you write a lot of technical documents, especially those containing
formulas, you've probably used 
\href{http://freshmeat.net/projects/latex/}{\LaTeX}.  LaTeX is,
basically, a set of macros for 
\href{http://freshmeat.net/projects/tetex/}{\TeX}. TeX, in turn, is
a powerful typesetting system first developed by Donald Knuth. It has
become an important tool for people who prefer to look at a document
as series of logical units, leaving the actual presentation or layout
to the software. LaTeX was developed by Leslie Lamport to aid in the
writing of classes of documents such as journal articles, book
chapters, and even letters. LaTeX abstracts many of the nitty gritty
details of TeX, such as margin widths, line offsets, etc., allowing
the user to simply decide on a document class and leave the style and
format to the macros.



Numerous people have written macro packages that can be used with LaTeX.
These packages provide an enormous range of functions, from formatting
of citations to drawing Feynman diagrams. Together with features such as
automatic index generation and bibliographies (using the 
\href{http://www.ecst.csuchico.edu/~jacobsd/bib/formats/bibtex.html}{Bib\TeX}
package), they provide the technical writer with an extremely powerful
tool to create beautiful documents, concentrating on the logical flow,
rather than having to worry about the underlying details of formatting
and layout.


However, documents are not the only the things that need to be
written; many times, a presentation must be made. Under Linux, tools
such as \href{http://freshmeat.net/projects/koffice/}{KPresent} and
\href{http://freshmeat.net/projects/magicpoint/}{MagicPoint} exist, and, of
course, Windows users have 
\href{http://www.microsoft.com/office/powerpoint/default.asp}{MS
PowerPoint}. These are the traditional GUI tools. However, when you
have to make a presentation containing formulas, they seem a little
clunky, and you're stuck with whatever the package
provides. Furthermore, if your documents are written using LaTeX, it
would be nice if you could use those documents to generate slides for
a presentation.


TeX and LaTeX being the all-powerful pieces of software they are, this
is indeed possible. However, the problem with making presentations in
LaTeX is the large number of packages available to do so.  I've listed
a few of the packages available, but there are quite a few more which
I haven't mentioned.


\subsection{The slides class}

Part of the \LaTeX distribution, it defines the page
sizes, font sizes, etc. suitable for printing transparencies.
Though the resultant DVI file can be converted to a PDF,
there is no support for the various features of PDFs such
as slide transitions and hyperlinks. Also, the package
provides no defined slide styles (i.e., backgrounds, frames,
etc.).

\begin{description}
   \item[The 
      \href{http://freshmeat.net/projects/seminar/}{Seminar}
      package] 
      Developed by Timothy van Zandt, this is an extremely powerful
      set of macros with which you can develop presentations that
      take full advantage of the PostScript and PDF
      specifications. There are an extremely large number of
      options and commands available for this package, so the
      learning curve is a little steep.

      \item[The 
      \href{http://www.cs.berkeley.edu/~mdw/proj/texslides/}{PDFLatex} package]
      This package is specifically designed for
      converting LaTeX source files to the PDF format without having
      to go through the intermediate DVI stage. Using this
      package along with the 
      \href{http://freshmeat.net/projects/foiltex/}{FoilTeX},
      pdfslide, and 
      \href{http://freshmeat.net/projects/ppower4/}{PPower4}
      packages allows you to generate presentations as well.
      
      \item[ The \href{http://freshmeat.net/projects/prosper/}{Prosper} package]
      This is a set of macros which allows you to generate
      PostScript or PDF presentations. There are certain advantages
      of this package over the others. First, though it has a
      simple structure, it provides enough options to
      generate good-looking slides. All the features of a PDF
      document (such as transitions, overlays, etc.) are
      available. In addition, it is easy to generate different slide
      styles, a la PowerPoint. Of course, you still have access to the
      full power of TeX, so you are free to extend your documents if
      you have the knowhow. For LaTeX beginners, however, Prosper
      encapsulates a lot of the details in an easy-to-use manner.
\end{description}

In this article, I'll be discussing the Prosper package in some
detail. You can find a good review of presentation tools for both PDF
and HTML formats 
\href{http://www.miwie.org/presentations/presentations.html}{here}.


\section{Prosper}

All LaTeX documents have a common basic structure. The first line
always defines the document type -- article, letter, chapter, or, in
this case, slides. After that comes the preamble. In the case of
Prosper, this is where you specify the title slide.  The next section
is the document proper. When using Prosper, this is where you define
the contents of successive slides. I'll cover the individual sections
of a document written with Prosper in detail, but the first step is to
install the package.

\subsection{Installation:}

As I mentioned above, the Prosper package provides a set of macros
which define functional elements of a presentation -- the slides, how
slides should transition, etc. To use the package, you will require
the <i>seminar</i>, <i>pstricks</i>, and <i>hyperref</i> packages
(which come with the standard TeX distribution on Red Hat). To
generate the final output, you'll also need <i>dvips</i>,
<i>GhostScript</i>, and <i>ps2pdf</i>. After downloading the tarball,
extract it into a directory. To make use of the package and associated
style files, you can place the required files (prosper.cls, the style
file that you are using, and any associated images, such as for
bullets) into the directory that contains your LaTeX
document. However, a neater method is to put the Prosper directory
into your TEXINPUTS environment variable:

<p>

<pre>
~: export TEXINPUTS=~/src/tex/Prosper:$TEXINPUTS
</pre>

<p>

(Where ~/src/tex/Prosper is the directory into which you extracted the
Prosper files.) That completes your installation.

<h3>The <i>prosper</i> Document Class</h3>

<p>

To make a presentation using the Prosper package, you need to specify
it in your \documentclass (you can also specify it in a \usepackage
command in the preamble).  Thus, the first line in the LaTeX file
should be of the form:

<p>

<pre>
\documentclass[ OPTIONS ]{prosper}
</pre>

<p>

There are several options that can be specified to the package.  You
can read about all the options in detail in the documentation that
comes with Prosper. I'll just give a brief overview of some of the
common and useful ones:

<p>

<dl>
  <dt>draft
  <dd>Compiles a draft version of the presentation,
      with figures replaced by bounding boxes.
  <dt>final
  <dd>Compiles a complete version of the presentation
      with figures and captions in their proper places.
  <dt>ps
  <dd>Compiles the LaTeX file to PostScript for printing
      purposes.
  <dt>pdf
  <dd>Compile the LaTeX file to a PDF format suitable for
      projectors.
</dl>

<p>

Another important option to specify is which presentation style to
use. Prosper comes with several styles, and new styles can easily be
made with a little knowledge of the <i>pstricks</i> package.

<p>

There are also options to specify slide background colors, slide
numbers, etc. In general, unless you require black and white slides
(e.g., for printing purposes), you won't need to set any color options
in the \documentclass; the style files will manage them for you.

<h3>The Preamble</h3>

<p>

The next section is the preamble, the part between \documentclass and
\begin{document}. In this section, you should specify the contents of
the title page and some options (such as logos and slide captions)
that can be applied to all the slides. The normal LaTeX macros have
been redefined to generate the title and associated text with proper
font sizes, etc. Some of the macros available for designing the title
slide include:

<p>

<ul>
  <li>\title
  <li>\subtitle
  <li>\author
  <li>\email
  <li>\slideCaption (You can use this macro to put a caption
      at the bottom of each slide.)
  <li>\Logo (This allows you to place a logo on each slide at a
      specified position.)
  <li>\DefaultTransition (This defines the type of transition that
      should occur between slides.)
</ul>

<p>

Since the <i>hyperref</i> package is included by Prosper, you can use
the \href command to include mailto: links or direct hyperlinks to Web
pages in the above commands (and, of course, in the rest of your
document).  As in standard LaTeX, the title slide is generated by the
\maketitle command in the document body.

<h3>The <i>slide</i> Environment</h3>

<p>

The Prosper package defines the <i>slide</i> environment. This
represents the basic unit of a presentation (a single slide) and is
placed in the document body (i.e., after the \begin{document}
command).  Within a slide environment, all the usual LaTeX commands
may be used. Images, formulas, tables, footnotes, page structure
commands, etc. can all be used. The Prosper package does redefine the
<i>itemize</i> environment so that the text is no longer justified. It
also supplies images for the bullets.  Thus, a single slide containing
a bulleted list can be represented by the following LaTeX
source (alongside, you can see how the final PDF output for this
slide would look):

<p>

<table align="center" border="0" cellspacing="5" summary="}{
  <tr>
    <td>
        <pre>
\begin{slide}{The Title of the Slide}
\begin{itemize}
\item Item 1
\item Item 2
\item Item 3
\end{itemize}
\end{slide}
        </pre>
    </td>
        <td align="center}{
        
        \href{http://images.freshmeat.net/editorials/prosper/pic2.jpg}{<img src="http://images.freshmeat.net/editorials/prosper/pic2-thumbnail.jpg" border="1" width="255" alt="" height="197}{}
    </td>
  </tr>
</table>

<p>

The environment does not provide any means to divide the slide area
into columns or rows; it simply provides a rectangular display area
(the dimensions of which may vary from style to style). However, using
the <i>minipage</i> environment, it is very easy to make a two-column
slide. For example, the following would create a slide with a picture
in one column and a bulleted list in the other:

<p>

<table align="center" border="0" cellspacing="5" summary="}{
  <tr>
    <td>
        <pre>
\begin{slide}{Another Example Slide}
\begin{minipage}{4cm}
\epsfig{file=./picture.eps}
\end{minpage}
\begin{minipage}{7cm}
\begin{itemize}
\item Item 1
\item Item 2
\item Item 3
\end{itemize}
\end{minipage}
\end{slide}
        </pre>
    </td>
        <td align="center}{
        
        \href{http://images.freshmeat.net/editorials/prosper/pic1.jpg}{ <img src="http://images.freshmeat.net/editorials/prosper/pic1-thumbnail.jpg" border="1" width="255" alt="" height="197}{}
    </td>
  </tr>
</table>

<p>

Prosper also defines some commands which are allowed to appear in a
<i>slide</i> environment. Examples include:

<dl>
  <dt>\FontTitle
  <dd>Defines the font to be used in the slide title
  <dt>\FontText
  <dd>Defines the font to be used in the slide text
  <dt>\fontTitle
  <dd>Writes its argument as the slide title
  <dt>\fontText
  <dd>Writes its argument as the slide text
</dl>

<p>

In general, the above macros are not used when writing a
presentation. They are, however, useful when you create slide styles
of your own.

<h3>Page Transitions</h3>

<p>

An important command is \PDFtransition, which can be used to specify
how the current slide should appear. However, the usual way to specify
a slide transition for a specific slide is to put the transition mode
into the \begin{slide} command as:

<p>

<pre>
\begin{slide}[Glitter]{Slide Title}
</pre>

<p>

The Prosper package supports several types of transitions:

<ul>
  <li>Split
  <li>Blinds
  <li>Box
  <li>Wipe
  <li>Dissolve
  <li>Glitter
  <li>Replace (the default)
</ul>

<p>

The above transition modes provide you with ample opportunity to make
flashy presentations (if that's what you're into :). You can see a PDF
which displays each of the transitions 
\href{http://images.freshmeat.net/editorials/prosper/trans.pdf}{here}.

<h3>Overlays</h3>

<p>

A very useful feature of computer-based presentations is the ability
to make overlay slides so parts of the same slide will appear at
different times. Prosper provides commands to implement this in a very
simple fashion.  The \overlay command is used to specify that a given
\slide environment will consist of a sequence of overlays. You must
specify the number of overlays that make up the slide.  
There are several commands that can be used to specify exactly what
material should appear on which slide within an overlay:

<dl>
  <dt>\fromSlide{p}{<i>material</i>}
  <dd>Puts <i>material</i> on
      slides p to the end of the overlay.
  <dt>\onlySlide{p}{<i>material</i>}
  <dd>Puts <i>material</i>
      only on slide p.
  <dt>\untilSlide{p}{<i>material</i>}
  <dd>Puts <i>material</i>
      on all slides from the first to the p<sup>th</sup>.
</dl>

<p>

There are three macros analogous to the above (obtained by
capitalizing the first letter) which cause all material after the
occurrence of the macro to be included (rather than specifically
defining <i>material</i>). The macros in the above list also have
starred counterparts (i.e., \fromSlide*, etc.).  These versions are
useful when the successive overlays need to replace previous
overlays. Below, I've provided an example of a slide that consists of
several overlays and uses the <i>itemstep</i> environment to allow an
itemized list to progress through successive overlays. Alongside is an
animation of how the PDF version of the slide would look:

<p>

<table align="center" cellspacing="5" border="0" summary="}{
  <tr>
    <td>
        <pre>
\overlays{5}{
\begin{slide}{The Effects of Power}
\begin{tabular}{rc}
\begin{minipage}{4cm}
\onlySlide*{1}{\epsfig{file=stage1.eps}}
\onlySlide*{2}{\epsfig{file=./stage2.eps}}
\onlySlide*{3}{\epsfig{file=./stage3.eps}}
\onlySlide*{4}{\epsfig{file=./stage4.eps}}
\onlySlide*{5}{\epsfig{file=./stage5.eps}}
\end{minipage} &
\begin{minipage}{6cm}
\begin{itemstep}
\item Alignment
\item Deformation
\item Coulomb explosion
\item X-ray emission
\item Nuclear reaction
\end{itemstep}
\end{minipage}
\end{tabular}
\end{slide}}
        </pre>
    </td>
    <td>
        \href{http://images.freshmeat.net/editorials/prosper/ol.gif}{<img src="http://images.freshmeat.net/editorials/prosper/ol.gif" width="255" border="1" alt="" height="197}{}
    </td>
  </tr>
</table>

<p>

An important point to note about the overlay commands is that they are
only valid when the Prosper package is used with the <i>pdf</i>
option. However, the package does provide a set of macros:

<ul>
  <li>\PDForPS{<i>ifpdf</i>}{<i>ifps</i>}
  <li>\onlyInPS{<i>material</i>}
  <li>\onlyInPDF{<i>material</i>}
</ul>

which allow you to include different material depending on whether
the LaTeX document is compiled in PS or PDF mode. An example of the
use of these macros would be:

<p>

<pre>
\overlays{3}{
\begin{slide}{An Example Slide}
\onlySlide*{1}{\epsfig=./pic1.eps}
\onlySlide*{2}{\epsfig=./pic2.eps}
\onlySlide*{3}{\epsfig=./pic3.eps}
\onlyInPS{\epsfig=./epspic.eps}
\end{slide}}
</pre>

<p>

If the snippet were converted to a PDF, we would get a slide which
would successively display pic1.eps, pic2.eps, and pic3.eps. If it
were compiled to PS format, the slide would only contain the image
epspic.eps.

<h3>Presentation Styles</h3>

<p>

The Prosper package comes with several style files. Essentially, these
provide predefined background colors and patterns, title fonts, bullet
styles, etc. You can easily change the look of your presentation by
including a different style file. Which style to use is specified in
the \documentclass. Below, you can see slides generated using the
different slide styles.

<p>

<table align="center" border="0" cellspacing="2" summary="}{
  <tr>
    <td>
        \href{http://images.freshmeat.net/editorials/prosper/style-default.jpg}{<img src="http://images.freshmeat.net/editorials/prosper/style-default-thumbnail.jpg"
        border="1" width="255" alt="" height="197}{}
        <br><b><div align="center}{Default</div></b>
    </td>
    <td>
        \href{http://images.freshmeat.net/editorials/prosper/style-alien.jpg}{<img src="http://images.freshmeat.net/editorials/prosper/style-alien-thumbnail.jpg"
        border="1" width="255" alt="" height="197}{}
        <br><b><div align="center}{Alienglow</div></b>
    </td>
    <td>
        \href{http://images.freshmeat.net/editorials/prosper/style-autumn.jpg}{<img src="http://images.freshmeat.net/editorials/prosper/style-autumn-thumbnail.jpg"
        border="1" width="255" alt="" height="197}{}
        <br><b><div align="center}{Autumn</div></b>
    </td>
  </tr>
  <tr>
    <td>
        \href{http://images.freshmeat.net/editorials/prosper/style-azure.jpg}{<img src="http://images.freshmeat.net/editorials/prosper/style-azure-thumbnail.jpg"
        border="1" width="255" alt="" height="197}{}
        <br><b><div align="center}{Azure</div></b>
    </td>
    <td>
        \href{http://images.freshmeat.net/editorials/prosper/style-blends.jpg}{<img src="http://images.freshmeat.net/editorials/prosper/style-blends-thumbnail.jpg"
        border="1" width="255" alt="" height="197}{}
        <br><b><div align="center}{Blends</div></b>
    </td>
    <td>
        \href{http://images.freshmeat.net/editorials/prosper/style-contemp.jpg}{<img src="http://images.freshmeat.net/editorials/prosper/style-contemp-thumbnail.jpg"
        border="1" width="255" alt="" height="197}{}
        <br><b><div align="center}{Contemporain</div></b>
    </td>
  </tr>
  <tr>
    <td>
        \href{http://images.freshmeat.net/editorials/prosper/style-db.jpg}{<img src="http://images.freshmeat.net/editorials/prosper/style-db-thumbnail.jpg"
        border="1" width="255" alt="" height="197}{}
        <br><b><div align="center}{Dark Blue</div></b>
    </td>
    <td>
        \href{http://images.freshmeat.net/editorials/prosper/style-frames.jpg}{<img src="http://images.freshmeat.net/editorials/prosper/style-frames-thumbnail.jpg"
        border="1" width="255" alt="" height="197}{}
        <br><b><div align="center}{Frames</div></b>
    </td>
    <td>
        \href{http://images.freshmeat.net/editorials/prosper/style-lb.jpg}{<img src="http://images.freshmeat.net/editorials/prosper/style-lb-thumbnail.jpg"
        border="1" width="255" alt="" height="197}{}
        <br><b><div align="center}{Lignes Bleues</div></b>
    </td>
  </tr>
  <tr>
    <td></td>
        <td align="center}{
        \href{http://images.freshmeat.net/editorials/prosper/style-ng.jpg}{<img src="http://images.freshmeat.net/editorials/prosper/style-ng-thumbnail.jpg"
        border="1" width="255" alt="" height="197}{}
        <br><b><div align="center}{Nuance trois</div></b>
    </td>
    <td></td>
  </tr>
</table>

<p>

It should be noted that all the styles do not provide the same display
area for the actual slide material. You can see this in some of the
slide examples above. If you decide to change the slide style of
your presentation, you might need to tweak things such as spacing
(\hspace, \vspace, etc.) or line lengths, etc. Furthermore, if a given
style does not really suit your taste, it is possible to make
modifications such as font type, colors, etc. using the Prosper
macros, rather than digging into the source of the style in question.

<p>

Assuming you're comfortable with the <i>pstricks</i> package,
designing a new slide is made easier by a number of macros defined by
Prosper. You have access to a number of boolean macros which allow you
to include features depending on the current environment (PDF or PS,
color or black &amp; white, etc.). The main macro that Prosper
provides to design a new style is the \NewSlideStyle command. After
designing the style, you need to tell Prosper the details, such as how
much display area you are providing, where it should be located, etc.,
using this macro.

<h3>Processing the LaTeX File</h3>

<p>

At this point, you should be able to write your presentation. The last
step is to convert the LaTeX source to a PDF file. The steps involved
are pretty simple:

<ol>
  <li>latex file.tex
  <li>dvips -Ppdf -G0 file.dvi -o file.ps
  <li>ps2pdf -dPDFsettings=/prepress file.ps file.pdf
</ol>

<p>

Two points to note:

<ul>
  <li>The -G0 parameter passed to dvips is used to get around a bug
      in GhostScript which converts the "f" character to a pound sign
      in the final PDF.
  <li>The -dPDFsettings parameter for ps2pdf is used to prevent
      downsampling of EPS images when they are converted to PDF.
      Without this switch, EPS graphics in the final PDF look very
      fuzzy, especially when viewed with a projector.
</ul>

<h3>Miscellaneous Features</h3>

<ul>
  <li>Since Prosper includes the <i>hyperref</i> package by
      default, you can easily set links and targets within your
      presentation with the \hyperlink and \hypertarget commands to
      enable easy navigation.
  <li>PowerPoint allows you to embed animations within a
      presentation. This is also possible when using Prosper, since it
      uses the <i>hyperref</i> package.  To embed an MPEG movie, you can
      include the following code snippet:
      <p>
      
      <pre>
\href{run:movie.mpg}{Click here to view the movie}
      </pre>
      
      <p>
      
      Two points to note:
      
      <ul>
        <li>Viewing the movie depends on Acrobat Reader being able
            to run the viewing program. This can be set by making sure
            you have an entry in your .mailcap file for the filetype you
            want to play.
        <li>The resultant movie plays in its own window; it is
            not possible to actually "embed" the movie in the
            presentation itself (at least under Linux).
      </ul>

      <p>
      
      Using this technique, you could run any type of file (assuming
      you have a program to handle it) or even executables like shell
      scripts, etc.

  <li>You may want to convert your PDF presentation to an HTML
      slideshow. This is possible using the program
      
      \href{http://jijo.cjb.net/code/python/python.html}{pdf2htmlpres.py}.
      It can use the <tt>convert</tt> program from the 
      \href{http://freshmeat.net/projects/imagemagick/}{ImageMagick}
      suite or
      GhostScript directly to convert the PDF slides to a series of
      JPGs (or GIFs or PNGs) and generate HTML pages to form a
      slideshow.
</ul>

<h3>Conclusion</h3>

<p>

I hope I've been able to convey some of the features and benefits that
the Prosper package provides. Granted, for a person who doesn't use
LaTeX, a GUI alternative would be easier. But for all the TeXnicians
out there, the Prosper package allows you to generate well-designed
and stylish slides efficiently, at the same time allowing the
knowledgeable user to extend the package using predefined macros and
pure TeX.

<p>

The Prosper community has a very useful mailing list which can be
accessed at the Prosper 
\href{http://freshmeat.net/projects/prosper/}{Web site}. The
Prosper tarball contains comprehensive documentation explaining the
available commands and macros provided by the package. It also
includes a document displaying the capabilities of the package. The
LaTeX sources of these documents are the best way to learn how to use
the various features of Prosper.

<p>

I, for one, have finally been able to get rid of MS PowerPoint and use
Prosper to develop all my presentations. Using this package, I'm able
to create presentations which rival those produced by more popular GUI
packages and which can be viewed with the very common Acrobat Reader
(and converted to <i>clean</i> HTML when required!).  You can take a
look at presentations I've made using prosper on 
\href{http://jijo.cjb.net/writing/pres/pres.html}{my Web site}

<p>

For all LaTeX users, I strongly recommend taking a look at Prosper.

</div>


<div class="article-author-bio}{
  <h2>Author's bio</h2>
  <p>

\href{http://jijo.cjb.net/}{Rajarshi Guha} is a grad student
working on computer-aided chemistry. When not studying and porting
ancient (and twisted) F77 code, he fiddles around with Python
programming and the Beowulf in his lab, and daydreams about marrying
his fiancee.

</div>


<div class="insetsize}{
  <div class="meta clearfix}{
    <ul class="tagList}{<li class="article-tags}{\href{/tags/tutorials/articles" rel="tag}{Tutorials}</li></ul>
  </div>
</div>


<div class="related-projects insetsize}{
  <h2>Related projects</h2>
  <ul class="clearfix}{
    
      <li>\href{/projects/imagemagick}{ImageMagick}</li>
    
      <li>\href{/projects/koffice}{KOffice}</li>
    
      <li>\href{/projects/latex}{LaTeX}</li>
    
      <li>\href{/projects/ppower4}{PPower4}</li>
    
      <li>\href{/projects/prosper}{Prosper}</li>
    
      <li>\href{/projects/tetex}{teTeX}</li>
    
  </ul>
</div>

\end{document}