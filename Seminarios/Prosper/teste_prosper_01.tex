\documentclass[pdf,letterpaper,
%autumn,
%azure,
%rico,
%alienglow,
%contemporain,
darkblue,
%default,
%frames,
%gyom, %legalzinho
%lignesbleues,
%nuancegris,
%pascal,
%troispoints,
slideColor,colorBG]{prosper} %Aqui estamos falando que vamos usar o prosper com suporte a cor.
\usepackage[english,portuges,brazilian,brazil]{babel} %escolha dos pacotes de linguagem, para acentua��o hifeniza��o entre outras coisas.
\usepackage[T1]{fontenc}
%\usepackage[latin1]{inputenc} %permite acentos
\usepackage{amsthm,amsfonts,amssymb,amsmath} %pacotes para escrita matem�tica
\usepackage{graphics} %pacote para inclus�o de figuras


\title{Um exemplo de Slide}
\subtitle{para o VOL}
\author{Salviano}
\email{salvianoleao@gmail.com}
\date{22/11/2013}
\institution{Instituto de F\'{\i}sica -- UFG}


\begin{document} %inicio dos slides


\maketitle %cria um slide de "capa"


\begin{slide}{C\'{o}digo}
Diz-se que $C$ � um $[n,k,d]-$c\'{o}digo, se ele \'{e} um subespa\c{c}o vetorial 
de $\mathbb{F}_q^n$ com dimens\~{a}o $k$ e dist\^{a}ncia m\'{\i}nima $d$.
\end{slide}


\overlays{3}{
\begin{slide}{C\'{o}digo}
    \defna{C\'{o}digo}{
    Chamamos de c\'{o}digo linear a um $\mathbb{F}_q-$subespa\c{c}o vetorial 
    de $\mathbb{F}_q^n$, onde $\mathbb{F}_q$ \'{e} um corpo finito com $q$ 
    elementos que aqui chamamos de alfabeto.\
                }
    \fromSlide{2}{
         \defna{Dist\^{a}ncia de Hamming}{
         Sendo $x, y \in \mathbb{F}_q^n$ definimos dist\^{a}ncia de $x$ \`{a} $y$ como:
   $$d(x,y) = \lbrace i ; 1leqslant i leqslant n, x_i eq y_i \rbrace .$$\
                        }
                }
   \fromSlide{3}{
         \defna{ 
         Dado $x \in \mathbb{F}_q^n$, definimos o peso de $x$ como sendo: $w(x)=d(x,0).$
                        }
                }
        \end{slide}
}
%*******************************2
\end{document}
